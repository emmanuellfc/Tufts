\documentclass{beamer}
\setbeamertemplate{caption}[numbered]
\usetheme{CambridgeUS}
 % \usetheme{Berlin}
% \usetheme{Copenhagen}
% \usetheme{AnnArbor}
% \usetheme{Antibes}
% \usetheme{Bergen}
\usepackage{graphicx} % Required for inserting images
\usepackage{tikz}
\usepackage{wrapfig}
\usepackage{tcolorbox}
\usepackage{caption}

\title[Coupling constant between NP and LC]{Mean field free-energy coupling parameter of nanoparticles and liquid crystal from molecular simulation of the isotropic-nematic transition.}

\author[]{J. Emmanuel Flores \inst{1}}

\institute[Journal Club, Spring 2024]{\inst{1} Tufts University, Medford, MA}

\date{\today}
% logo of my university
\titlegraphic{\includegraphics[width=4cm]{Figures/A&S_Hori_BK+BL.jpg}}

\AtBeginSection[] % Show Outline at the beginning of every new section.
{
    \begin{frame}
        \frametitle{Table of Contents}
        \tableofcontents[currentsection]
    \end{frame}
}

\begin{document}

\maketitle

\begin{frame}
    \frametitle{Outline}
      \tableofcontents
\end{frame}

\section{Introduction}
\subsection{Segregation of NP in a LC and continuum models.}

\begin{frame}{Self-Assembly}
    \begin{itemize}
        \item Disordered system $\rightarrow$ structure
        \begin{itemize}
            \item Consequence between interaction of the components.
            \item No external direction.
        \end{itemize}
        \item Examples
        \begin{itemize}
            \item Phospholipid bilayers
            \item Viral apsids
            \item Block copolymer gyroids
            \item Micro shells of Nanpoarticles (NPs)
        \end{itemize}
        \item Applications
        \begin{itemize}
            \item Drug delivery
            \item Photonic crystals 
            \item Biological systems.
        \end{itemize}
    \end{itemize}
\end{frame}

\begin{frame}{Self-assembly of NP in a LC}
    \begin{columns}
        \begin{column}{0.4\textwidth}
            \begin{itemize}
        \item Fluorescent NP
        \begin{itemize}
            \item Functionalized with mesogens.
            \item Soluble in 5CB
            \item Insoluble in 5CB
        \end{itemize}
        \item Micro shell formation
        \begin{itemize}
            \item Stable structures.
        \end{itemize}
    \end{itemize}
        
        \end{column}

        \begin{column}{0.6\textwidth}
            \begin{figure}
                \centering
                \includegraphics[scale = 0.28]{Figures/Linda_Experiment.png}
                \caption{A. L. Rodarte, et al. Soft  Matter (2015)}
                \label{fig:enter-label}
            \end{figure}
        \end{column}
        \end{columns}
\end{frame}

\begin{frame}{Continuum Models}
    \begin{itemize}
        \item Volume fraction of NPs: conserved field $\eta$.
        \item Order parameter of the liquid crystal (LC): non conserved field $S$.
        \item Free energy of coupling between $\eta$ and $S$
        \begin{equation}
            F = \frac{a(T-T^*)}{2}S^2 - \frac{\omega}{3}S^3 + \frac{u}{4}S^4 + B\eta^2 + \frac{q}{2}\eta S^2
        \end{equation}
    \end{itemize}
\end{frame}

\begin{frame}{Determination of $g$}
    \begin{itemize}
        \item $g$ $\rightarrow$ coupling between fields.
        \item Parameter required for the theory.
        \item Hard to measure experimentally.
        \item Alternative: Molecular Dynamics (MD).
    \end{itemize}
\end{frame}

\section{Model and simulation method}
\subsection{Rigid chains of LJ: Molecular Dynamics.}

\begin{frame}{Model and Statistical Ensemble}
   \begin{columns}
        \begin{column}{0.5\textwidth}
        \begin{itemize}
            \item LC: rigid chain of 9 Lenard-Jones(LJ) spheres.
            \item NPs: spheres of LJ, bigger than LC chains.
            \item NPT ensemble.
            \item Gradual elevation of the pressure and temperature to generate initial states.
        \end{itemize}
        \end{column}
        \begin{column}{0.5\textwidth}
            \begin{figure}
                \centering
                \includegraphics[scale = 0.20]{Figures/Model.png}
                \caption{{\scriptsize a. Representative screenshot of the systems, b. Oyarzún et al. Molecular Physics (2016).}}
                \label{fig:enter-label}
            \end{figure}
        \end{column}
        \end{columns}
    
\end{frame}

\begin{frame}{HOOMD-blue}
HOOMD-blue is a general-purpose particle simulation toolkit, implementing molecular dynamics and hard particle Monte Carlo optimized for fast execution on both GPUs and CPUs.
        \begin{itemize}
            \item Python package.
            \item Active development.
            \item Easy instalation.
        \end{itemize}
        \begin{figure}
            \centering
            \includegraphics[scale = 0.15]{Figures/hoomd.png}
            %\caption{Caption}
            \label{fig:Hoomd Logo}
        \end{figure}
\end{frame}

\begin{frame}{Methodology}
    \begin{columns}
        \begin{column}{0.5\textwidth}
        \begin{itemize}
            \item Isobars studied: $P \in \{1.3, 1.5, 1.8, 2.3\}$
            \item Sigmas studied: $\sigma \in \{0, 2, 4, 5\}$
            \item Explicitly:
                \begin{itemize}
                    \item $\sigma = 0$ $\implies P_{\sigma_0} \in \{1.3, 1.5, 1.8, 2.3\}$
                    \item $\sigma = 2$ $\implies P_{\sigma_2} \in \{1.3, 1.5, 1.8, 2.3\}$
                    \item $\sigma = 4$ $\implies P_{\sigma_4} \in \{1.3, 1.5, 1.8, 2.3\}$
                    \item $\sigma = 5$ $\implies P_{\sigma_5} \in \{1.3, 1.5, 1.8, 2.3\}$
                \end{itemize}
        \end{itemize}
        
        \end{column}

        \begin{column}{0.6\textwidth}
            \begin{figure}
        \centering
        \includegraphics[scale=0.22
]{Figures/Simulated_Systems.png}
        \caption{{\scriptsize Points for simulation.}}
        \label{fig:enter-label}
    \end{figure}
        \end{column}
    \end{columns}
\end{frame}


\section{Results}
\subsection{I-N transition, coupling constant, entropy changes and latent heat.}

\begin{frame}{Isotropic-Nematic transition}
    \begin{itemize}
        \item The order parameter S decreases rapidly with increasing temperature above TIN
        \item Non linear Fit.
    \end{itemize}
    \begin{figure}
        \centering
        \includegraphics[scale = 0.25]{Figures/Transition.png}
        \caption{Order parameter as a function of temperature.}
        \label{fig:enter-label}
    \end{figure}
\end{frame}

\begin{frame}{Determination of $g$}
    \begin{columns}
        \begin{column}{0.5\textwidth}
        From Landau-de Gennes theory, we know that 
        $$g = -R\frac{aT_{c_0}}{\eta}$$
        So, using our results for R and η, along with the numericla values 
        $$a = 190 kPa/K (*),$$
        $$T_{c_0} = 313.8 (**),K$$
        it is possible to calculate the coupling constant for 8CB
        $$g = 53MPa$$
        \end{column}

        \begin{column}{0.5\textwidth}
        \begin{figure}
            \centering
            \includegraphics[scale = 0.30]{Figures/G_Det1.png}
            \caption{{\scriptsize Transition temperature as a function of the packing factor for isobars 1.3 (red), 1.5 (green), 1.8 (blue) and 2.3 (black).}}
            \label{fig:enter-label}
        \end{figure}
        \end{column}
    \end{columns}
\end{frame}

\begin{frame}{Dependence with the pressure}
    \begin{columns}
        \begin{column}{0.5\textwidth}
        \begin{itemize}
            \item The coupling “constant” depends on the pressure:  $g = g(P)$
            \item The I-N transition temperature decreases with increasing NPs concentration.
            \item The I-N transition temperature also depends on pressure, at fixed $\eta $.
        \end{itemize}
        \end{column}
        \begin{column}{0.5\textwidth}
        \begin{figure}
            \centering
            \includegraphics[scale = 0.20]{Figures/P_Dep.png}
            \caption{{\scriptsize a. Nematic isotropic transition temperature for different diameters of NPs., b. Volume fraction for different diameters of NPs.}}
            \label{fig:enter-label}
        \end{figure}
        \end{column}
    \end{columns}
\end{frame}

\begin{frame}{Latent heat per particle}
    \begin{columns}
        \begin{column}{0.5\textwidth}
        \begin{itemize}
            \item The latent heat per particle associated with the I-N transition increases linearly with pressure for the pure system and each of the other systems.
        \end{itemize}
        \end{column}
        
        \begin{column}{0.5\textwidth}
        \begin{figure}
            \centering
            \includegraphics[scale = 0.20]{Figures/L_Heat.png}
            \caption{{\scriptsize a. Values of latent heat per particle associated with the I-N transition, b. Graph of the latent heat per particle associated with the I-N transition for the pure system (red), (green), (blue) and black.}}
            \label{fig:enter-label}
        \end{figure}
        \end{column}
    \end{columns}
\end{frame}

\begin{frame}{Entropy of I-N transition}
    \begin{columns}
        \begin{column}{0.5\textwidth}
        \begin{itemize}
            \item The entropy change per particle associated with the I-N transition is independent of pressure, but is proportional to volume.
        \end{itemize}
        \end{column}
        
        \begin{column}{0.5\textwidth}
        \begin{figure}
            \centering
            \includegraphics[scale = 0.22]{Figures/Entropy.png}
            \caption{{\scriptsize a. Values of the change in entropy per particle associated with the transition, b. Change in entropy per particle associated with the transition.}}
            \label{fig:enter-label}
        \end{figure}
        \end{column}
    \end{columns}
\end{frame}

\section{Conclusions}
\begin{frame}{Conclusions}
    \begin{itemize}
        \item With this methodology it is possible to find the coupling parameter between NPs and mesogens, calculate the latent heat of transition and the entropy associated with this change (because HOOMD can be used on GPUs it makes the method efficient).

        \item The coupling parameter is strongly related to pressure.

        \item The observed linear dependence of $\eta$ on the transition temperature indicates that the single-particle theoretical description might be suitable for interpreting the associated free energy coupling term via the chemical potential.

        \item The dependence of $g$ on $P$ can lead to the formation of spatially periodic NP structures upon standing acoustic waves or passing shock waves.
    \end{itemize}
\end{frame}

% \subsection{Physics and PDE}
% \begin{frame}{Physics and PDE}
%     \begin{columns}
%         \begin{column}{0.6\textwidth}
%             \begin{itemize}
%                 \item<1-> Almost all of the physical laws are written in the language of PDEs.
%                 \item<2-> The huge complexity of these equations make them difficult(or impossible) to solve analytically.
%                 \item<3-> Numerical Methods:
%                     \begin{itemize}
%                         \item<4-> Finite Differences
%                         \item<5-> Finite Element (Volume) Methods.
%                         \item<5-> Spectral Methods (Chebyshev polynomials).
%                     \end{itemize}
%             \end{itemize}
%         \end{column}
%         \begin{column}{0.37\textwidth}
%             \begin{tcolorbox}[title= Maxwell's Equations]
%             \begin{itemize}
%                 \item $\nabla\cdot \mathbf{E} =\rho / \epsilon_0,$
%                 \item $\nabla\times \mathbf{E} = 0, $
%                 \item $\nabla\cdot \mathbf{B} = 0,$
%                 \item $\nabla\times \mathbf{B} = \mu_0\mathbf{J}.$
%             \end{itemize}
%             \end{tcolorbox}
%         \end{column}
%     \end{columns}    
% \end{frame}



% \section{Problem studied: Coaxial Cable}

% \subsection{Statement of the Problem}

% \begin{frame}{Statement of the Problem}
% \begin{columns}
%     \begin{column}{0.6\textwidth}
%     \begin{itemize}
%         \item Coaxial Cylinders
%         \begin{itemize}
%             \item Radial symmetry
%             \item Analytic solution is nonlinear
%         \end{itemize}
%     \end{itemize}
%     \end{column}
%     \begin{column}{0.4\textwidth}  %%<--- here
%         \begin{center}
%             \begin{tikzpicture} [scale=0.40]    
%             \draw[help lines, color=gray!30, dashed] (-4.4,-4.4) grid (4.4,4.4);
%             \draw[->,ultra thick] (-4.5,0)--(4.5,0) node[right]{\small $x$};
%             \draw[->,ultra thick] (0,-4.5)--(0,4.5) node[above]{\small $y$};
%             \draw [thick] circle [radius=2.0];
%             \draw [thick] (0,0) circle [radius=4];
%             \draw[-, rotate around={-45:(0,0)}] (0,0) -- (2,0)  node [pos=1.45] {\small $a$, $V_a$};
%             \draw[-, rotate around={20:(0,0)}] (0,0) -- (4.,0) node [pos=0.7, above] {\small $b$, $V_b$};
%             \draw[fill=black](0,0) circle (1 pt) node [anchor=south east] {\small $O$};
%             \end{tikzpicture}
%             \captionof{figure}{Two dimensional schematic projection of the coaxial problem.}
%          \end{center}
%     \end{column}
% \end{columns}
% \end{frame}

% \subsection{Analytical Solution}

% \begin{frame}{Analytical Solution}
% \begin{columns}
% \begin{column}{0.6\textwidth}
% \begin{itemize}
%     \item<1-> Laplacian in Cylindrical Coordinates
%     \begin{equation}
%     \frac{1}{r}\frac{\partial}{\partial r}\left(r\frac{\partial V}{\partial r}\right) + \frac{1}{r^2}\frac{\partial^2 V}{\partial \theta^2} + \frac{\partial^2 V}{\partial z^2} = 0,
%     \end{equation}
%     \item<2-> Simplification by symmetry
%     \begin{equation}
%     \frac{1}{r}\frac{\partial}{\partial r}\left(r\frac{\partial V}{\partial r}\right)=0,
%     \end{equation}
%     \item<3-> Analytical Solution
%     \begin{equation}
%     V(r) = \frac{V_b - V_a}{\ln(b/a)}\ln\left(\frac{r}{a}\right) + V_a.
%     \label{eq:coax_pot}
%     \end{equation}
% \end{itemize}
% \end{column}
%     \begin{column}{0.4\textwidth}  %%<--- here
%         \begin{center}
%             \begin{tikzpicture} [scale=0.40]    
%             \draw[help lines, color=gray!30, dashed] (-4.4,-4.4) grid (4.4,4.4);
%             \draw[->,ultra thick] (-4.5,0)--(4.5,0) node[right]{\small $x$};
%             \draw[->,ultra thick] (0,-4.5)--(0,4.5) node[above]{\small $y$};
%             \draw [thick] circle [radius=2.0];
%             \draw [thick] (0,0) circle [radius=4];
%             \draw[-, rotate around={-45:(0,0)}] (0,0) -- (2,0)  node [pos=1.45] {\small $a$, $V_a$};
%             \draw[-, rotate around={20:(0,0)}] (0,0) -- (4.,0) node [pos=0.7, above] {\small $b$, $V_b$};
%             \draw[fill=black](0,0) circle (1 pt) node [anchor=south east] {\small $O$};
%             \end{tikzpicture}
%             \captionof{figure}{Two dimensional schematic projection of the coaxial problem.}
%          \end{center}
%     \end{column}
%     \end{columns}
% \end{frame}

% \section{Computational Setup}
% \subsection{General Finite Differences Method}
% \begin{frame}{Finite Differences Methods}
% \begin{columns}
%     \begin{column}{0.6\textwidth}
%         \begin{itemize}
%             \item <1-> How can we approximate the Laplacian operator?
%             \begin{equation*}
%                 \nabla^2 \xrightarrow{Discretize} \;\;\; ???
%             \end{equation*}
%             \item <2-> Use a quotient to approximate partial derivatives
%             \begin{equation}
%                 \frac{\partial^2 V}{\partial x^2} \approx \frac{1}{2} \frac{\Delta V}{(\Delta x)^2}
%             \end{equation}
%             \item<2-> We can model a system with a series of these \textbf{finite differences} to approximate a partial differential operator
%         \end{itemize}
%     \end{column}
%     \begin{column}{0.4\textwidth}
%         % \begin{figure}[h]
%         %     \centering
%         %     \includegraphics[width=0.95\textwidth]{Figures/Landau_FDM.png}
%         %     \caption{   (TODO: Cite me.)}
%         %     \label{fig:landau_fdm}
%         % \end{figure}
% %         \begin{tikzpicture}
% % [
% %   xscale=0.8,
% %   yscale=0.8,
% %   grid/.style={draw,thin,gray!50},
% %   node/.style={draw,circle,fill=black,inner sep=1pt}
% % ]

% % % Grid lines
% % \draw[grid] (0,0) grid (4.5,4.4);
% % \draw[help lines, color=gray!30, dashed] (-0.5,-0.5) grid (4.4,4.4);
% % \draw[->,ultra thick] (-0.25,-0.25)--(4.5,-0.25) node[right]{$x$};
% % \draw[->,ultra thick] (-0.25,-0.26)--(-0.25,4.5) node[above]{$y$};

% % % Grid points
% % \foreach \x in {0,1,2,3,4} {
% %   \foreach \y in {0,1,2,3,4} {
% %     \node [node] at (\x,\y) {};
% %   }
% % }

% % \end{tikzpicture}

% \begin{tikzpicture}
% [
%   xscale=0.9,
%   yscale=0.9,
%   grid/.style={draw,thin,gray!50},
%   node/.style={draw,circle,fill=gray,inner sep=1pt}
% ]

% % Grid lines
% \draw[grid] (0,0) grid (4.5,4.4);
% \draw[help lines, color=gray!10, dashed] (-0.5,-0.5) grid (4.4,4.4);
% \draw[->,ultra thick] (-0.25,-0.25)--(4.5,-0.25) node[right]{$x$};
% \draw[->,ultra thick] (-0.25,-0.26)--(-0.25,4.5) node[above]{$y$};

% % Grid points
% \foreach \x in {0,1,2,3,4} {
%   \foreach \y in {0,1,2,3,4} {
%     \node [node] at (\x,\y) {};
%   }
% }
% \filldraw[red] (2,2) circle (2pt) node[anchor=south]{\tiny $(i,j)$};
% \filldraw[blue] (3,2) circle (2pt) node[anchor=north]{\tiny $(i+1,j)$};
% \filldraw[blue] (1,2) circle (2pt) node[anchor=north]{\tiny $(i-1,j)$};
% \filldraw[blue] (2,3) circle (2pt) node[anchor=south]{\tiny $(i,j+1)$};
% \filldraw[blue] (2,1) circle (2pt) node[anchor=north]{\tiny $(i,j-1)$};

% \end{tikzpicture}
%     \end{column}
% \end{columns}
% \end{frame}

% \begin{frame}{Generalizing Finite Differences}
% \begin{columns}
%     \begin{column}{0.6\textwidth}
%         \begin{itemize}
%             \item <1-> Express FDM on an arbitrary graph (mesh)
%             \item <1-> Every node wants to satisfy Laplace's equation,
%             \begin{equation}
%                 \nabla^2 V_i = 0
%             \end{equation}
%             \item <2-> Using a finite differences approximation, every node is informed by its neighbors
%             \item <2-> Fix potential of some nodes (boundary conditions), then generate a linear system of equations that can be solved
%         \end{itemize}
%     \end{column}
%     \begin{column}{0.4\textwidth}
%         \begin{figure}[h]
%             \centering
%             \includegraphics[width=0.95\textwidth]{Figures/Coax_GFDM_Connectivity.png}
%             \label{fig:coax_gfdm_connectivity}
%             \caption{Schematic of the GFDM approach to the COAX problem. Black nodes are boundaries, lines show connections generated by the GFDM algorithm.}
%         \end{figure}
%     \end{column}
% \end{columns}
% \end{frame}

% \section{Finite Element Method}
% \begin{frame}{FEM Approach}
%     \begin{columns}
%         \begin{column}{0.5\textwidth}
%             \begin{itemize}
%             \item<1-> Strong Form: function and its derivatives for a well posed problem.
%             \item<2-> What if we expand/relax the conditions?
%             \item<3-> Express the problem in a variational way
%             \item<4-> Approximate the solution in each element and interpolate solutions.
%             \end{itemize}
%         \end{column}

%         \begin{column}{0.4\textwidth}
%             \begin{itemize}
%                 \item<5-> Weak Form:
%                     \begin{equation*}
%                         \int_{\Omega}d\Omega( \nabla V\cdot\nabla \eta )= 0
%                     \end{equation*}
%                 \begin{figure}
%                     \centering
%                     \includegraphics[width=0.75  \textwidth]{Figures/Coaxial_Mesh-removebg-preview.png}
%                     \caption{Mesh for the coaxial cable problem.}
%                     \label{fig:Coax_Mesh}
%                 \end{figure}
%             \end{itemize}
%         \end{column}
%     \end{columns}
% \end{frame}

% \section{Results}

% \subsection{GFD and FEM}

% \begin{frame}{Coaxial Cable}
% \begin{figure}[h]
% \begin{columns}
%     \begin{column}{0.5\linewidth}
%         \centering\includegraphics[width=0.85\textwidth]{Figures/Coax_GFDM.png}
%     \end{column}
%     \begin{column}{0.5\linewidth}
%         \centering\includegraphics[width=1.0\textwidth]{Figures/PotentialFieldFEMCoaxial.png}
%     \end{column}
% \end{columns}
% \caption{
% Potential obtained for the coaxial cable. \textbf{Left}: GFD approach. \textbf{Right}: FEM approach. As we can see, we have a qualitatively good agreement between both approaches.}
% \label{fig:tCoaxial_Cable_VField}
% \end{figure}
% \end{frame}

% \begin{frame}{Comparison Between Both Approaches}
% \begin{figure}[h]
% \begin{columns}
%     \begin{column}{0.5\linewidth}
%         \includegraphics[width=0.95\textwidth]{Figures/ComparissonPotentialFEMGFD.png}
%     \end{column}
%     \begin{column}{0.5\linewidth}
%         \includegraphics[width=0.95\textwidth]{Figures/ComparissonPotentialFEMGFD_Zoom.png}
%     \end{column}
% \end{columns}
% \caption{Comparison between both approaches for the coaxial cable, for GFD we have an RMS = 0.084, whereas for FEM RMS = 0.008. \textbf{Left}: Plot for the whole domain of the radius. \textbf{Right}: Zoom on the zone where the solutions are most different between each other.}
% \label{fig:Comparison_GFD_FEM}
% \end{figure}
% \end{frame}

% \begin{frame}
% \begin{columns}
%     \begin{column}{0.5\linewidth}
%     \begin{itemize}
%         \item We tried running GFDM on the coaxial problem many times to see if error decreases with higher resolution
%         \item Low node count causes noise in the solution
%         \item High node count shows a more stable and lower error solution
%     \end{itemize}
%     \end{column}
%     \begin{column}{0.5\linewidth}
%     \begin{figure}[h]
%         \centering
%         \includegraphics[width=0.95\textwidth]{Figures/Coax_GFDM_Nodes_Neighbors_RMS.png}
%         \caption{RMS error of many independent GFDM solutions for the coaxial problem.}
%         \label{fig:coax_gfdm_nodes_neighbors}
%     \end{figure}
%     \end{column}
% \end{columns}
% \end{frame}

% % \begin{frame}{FEM: Comparison with the potential field}
% %     \begin{figure}
% %         \centering
% %         \includegraphics[scale=0.4]{Figures/ComparissonPotentialFEMCoaxial.png}
% %         \caption{Potential field as function of the radial distance: Comparison between the analytical solution and the approximate solution using FEM.}
% %         \label{fig:FEM_V_Sol}
% %     \end{figure}
% % \end{frame}

% % \begin{frame}{GFD vs FEM}
% %     \begin{figure}
% %         \centering
% %         \includegraphics[scale = 0.5]{Figures/ComparissonPotentialFEMGFD.png}
% %         \caption{Caption}
% %         \label{fig:enter-label}
% %     \end{figure}
% % \end{frame}

% \subsection{Complex Geometries}
% % \begin{frame}{Two rectangle bars}
% %     \begin{figure}
% %         \centering
% %         \includegraphics[scale = 0.4]{Figures/TwoBars.png}
% %         \caption{Two Bars}
% %         \label{fig:enter-label}
% %     \end{figure}
% % \end{frame}


% \begin{frame}{Two conducting bars}
% \begin{figure}[h]
% \begin{columns}
%     \begin{column}{0.5\linewidth}
%         \includegraphics[width=0.95\textwidth]{Figures/Bars_GFDM.png}
%     \end{column}
%     \begin{column}{0.5\linewidth}
%         \includegraphics[width=0.95\textwidth]{Figures/Bars_FEM.png}
%     \end{column}
% \end{columns}
% \caption{Two conducting bars each held at 10.0 V with the bounding box grounded. \textbf{Left}: GFDM solution (5822 nodes). \textbf{Right}: FEM solution.}
% \label{fig:two_bars}
% \end{figure}
% \end{frame}

% \begin{frame}{Error for two conducting bars}
% \begin{columns}
%     \begin{column}{0.5\linewidth}
%     \begin{itemize}
%         \item GFDM solution does well between the conductors
%         \item Outside the conductors, the solution is not as good
%         \item Solution is sensitive to irregularities in the mesh
%     \end{itemize}
%     \end{column}
%     \begin{column}{0.5\linewidth}
%         \begin{figure}[h]
%         \includegraphics[width=0.95\textwidth]{Figures/Bars_Error.png}
%         \caption{Error (GFDM - FEM) for the two conducting bars problem. RMS = 0.577}
%         \label{fig:two_bars_error}
%         \end{figure}
%     \end{column}
% \end{columns}
% \end{frame}

% \begin{frame}{Conducting dolphin}
% \begin{figure}[h]
% \begin{columns}
%     \begin{column}{0.5\linewidth}
%         \includegraphics[width=0.95\textwidth]{Figures/Dolfin_GFDM.png}
%     \end{column}
%     \begin{column}{0.5\linewidth}
%         \includegraphics[width=0.95\textwidth]{Figures/Dolfin_FEM.png}
%     \end{column}
% \end{columns}
% \caption{Ideally conducting dolphin held at 10.0 V with the bottom boundary held at 5.0 V and other boundaries grounded. \textbf{Left}: GFDM solution (2868 nodes). \textbf{Right}: FEM solution.}
% \label{fig:dolfin}
% \end{figure}
% \end{frame}

% \begin{frame}{Conducting dolphin error}
% \begin{columns}
%     \begin{column}{0.5\linewidth}
%     \begin{itemize}
%         \item Dolphin solution is not very good for GFDM compared to FEM
%         \item Reason is unknown, sparseness of points far from the dolphin may be causing a problem
%         \item Laplacian approximation failing may be the reason for poor solution throughout
%     \end{itemize}
%     \end{column}
%     \begin{column}{0.5\linewidth}
%         \begin{figure}[h]
%         \includegraphics[width=0.95\textwidth]{Figures/Dolfin_Error.png}
%         \caption{Error (GFDM - FEM) for the two conducting bars problem. RMS = 1.870}
%         \label{fig:dolfin_error}
%         \end{figure}
%     \end{column}
% \end{columns}
% \end{frame}

% \begin{frame}{Conducting python}
% \begin{figure}[h]
% \begin{columns}
%     \begin{column}{0.5\linewidth}
%         \centering
%         \includegraphics[width=0.80\textwidth]{Figures/Python_GFDM.png}
%     \end{column}
%     \begin{column}{0.5\linewidth}
%         \centering
%         \includegraphics[width=0.80\textwidth]{Figures/Python_FEM.png}
%     \end{column}
% \end{columns}
% \caption{Ideally conducting python held at 10.0 V with the bounding box grounded. \textbf{Left:} GFDM solution (2201 nodes). \textbf{Right:} FEM solution.}
% \label{fig:python}
% \end{figure}
% \end{frame}

% \begin{frame}{Conducting python error}
% \begin{columns}
%     \begin{column}{0.5\linewidth}
%     \begin{itemize}
%         \item Python solution performs well near the snake
%         \item Some "hotspots" of error where the solver did not perform well
%         \item Points of high error have boundaries that are near each other
%         \item Perhaps the solver does not do well for quickly changing potentials
%     \end{itemize}
%     \end{column}
%     \begin{column}{0.5\linewidth}
%         \begin{figure}[h]
%         \includegraphics[width=0.80\textwidth]{Figures/Python_Error.png}
%         \caption{Error (GFDM - FEM) for the two conducting bars problem. RMS = 1.013}        \label{fig:python_error}
%         \end{figure}
%     \end{column}
% \end{columns}
% \end{frame}

\section*{Bibliography}
\begin{frame}[allowframebreaks]{Bibliography}

\beamertemplatebookbibitems
\begin{thebibliography}{}
% \bibitem{NetworkX}HAGBERC, ARIC, SCHULT, DAN, S. \newblock \textquotedbl NetworkX.\textquotedbl{} \newblock Software Package.

% \bibitem{-}Harris, Charles R., et al. \newblock \textquotedbl Array programming with NumPy.\textquotedbl{} \newblock Nature 585.7825 (2020): 357-362.

% \bibitem{--}Landau, Rubin H., Manuel J. Páez, and Cristian C. Bordeianu. \newblock \textquotedbl Computational physics: Problem solving with Python\textquotedbl{} \newblock John Wiley & Sons, 2015.

% \bibitem{---}LSchlömer, N. \newblock \textquotedbl "pygalmesh: Python interface for CGAL’s meshing tools.\textquotedbl{} \newblock DOI: https://doi. org/10.5281/zenodo 5564818 (2021).
\bibitem{---} (*1)Faetti S and Palleschi V. 1984 The Journal of Chemical Physics 81 6254–6258.
\bibitem{---} (*2)de Mesquita O N 1998 Brazilian Journal of Physics 28 00–00.
\bibitem{---} (**)Palermo M F, Pizzirusso A, Muccioli L and Zannoni C 2013 The Journal of Chemical Physics 138 204901


\end{thebibliography}
\end{frame}
\end{document}
