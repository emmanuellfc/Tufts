\documentclass[12pt]{beamer}
\usepackage{amsmath}
\usetheme[progressbar=frametitle]{metropolis}
\title{Position and the Free Particle}
\author{Emmanuel Flores}
\institute{Advanced Mathematical Methods, \\Tufts University}

\begin{document}
\maketitle
%%%%%%%%%%%%%%%%%%%%%%%%%
\begin{frame}{Contents}
	\setbeamertemplate{section in toc}[sections numbered]
	\tableofcontents
\end{frame}
%%%%%%%%%%%%%%%%%%%%%%%%%
\section[The Position Operator]{The Position Operator}
%%%%%
\begin{frame}{Preamble}
\begin{itemize}
	\item For a free particle, the Hamiltonian commutes with the momentum operator, i.e $[p,H] = 0$ as a result momentum is conserved and under time evolution eigenstates do not change.
	\item Under Fourier transformations position and momentum can be interchanged. Even more, position eigenstates will be $\delta$ functions (distributions), and now $[q,H]\neq0$
\end{itemize}
\end{frame}
%%%%%
\begin{frame}{The Position Operator}
Given a Hilbert Space $\mathcal{H}$	, we define the position operator by the eigenvalue problem
\begin{displaymath}
Q\psi(q) = q\psi(q)  
\end{displaymath}
\begin{itemize}
	\item It has the same issues as P $\implies $ we need to relax the space of admisible functions (distributions)
\end{itemize}
\end{frame}
%%%%%
\begin{frame}{Interdule 1: Distributions}
Also known as Schwartz distributions or generalized functions, which are defined as continuous linear functionals on a space of infitely differentiable test functions.
\begin{itemize}
	\item PDE \& Weak Solutions: enable the discovery of solutions that wouldn't exist in the classical sense, known as weak solutions.
	\item Enable construction of Sobolev Spaces: useful for analyzing the regularity of PDEs and their solutions.
	\item Physical motivation: singular initial conditions, problems dealing with discountinupus functions (etc).
\end{itemize}
\end{frame}
%%%%%
\begin{frame}{The Position Operator}
By the following relationship 
\begin{displaymath}
  \int_{\infty}^{\infty} q\delta(q-q^{\prime})f(q)dq = \int_{\infty}^{\infty} q^{\prime}\delta(q-q^{\prime})f(q)dq,
\end{displaymath}
it follows that 
\begin{displaymath}
  q\delta(q-q^{\prime}) = q^{\prime}\delta(q-q^{\prime}),
\end{displaymath}
in the sense of distributions. Thus; $\delta(q-q^{\prime})$ is eigenfunction of Q with eigenvalue $q^{\prime}$.
\end{frame}
%%%%%
\begin{frame}{The Position Operator}
$Q$ and $P$ do not commute, in fact 
\begin{displaymath}
  [Q,P]=i\hbar,
\end{displaymath}
And even more; $H$ commutes with $P$, but $P$ does not commute with $Q$ $\implies$ $Q$ does not commute with $H$, thus $Q$ is not conserved.
\end{frame}
%%%%%
\begin{frame}{Interlude 2: Spectral theorem}
Finite-Dimensional Case: for a Hermitian matrix, this theorem states that it is unitarily diagonalizable. 

This is:  $\exists$ unitary matrix $U$ such that 
\begin{displaymath}
  	A = U\Lambda U^{\dagger},
\end{displaymath}
where $\Lambda$  is a diagonal matrix containing the real eigenvalues of $A$, and the columns of  are the orthonormal eigenvectors of $A$.

Or in other words: every symmetric matrix can be diagonalized using orthogonal eigenvectors
\end{frame}
%%%%%
\begin{frame}{Interlude 2: Spectral theorem}
Infinite-Dimensional Case: for a self-adjoint operator  on a Hilbert space, the spectral theorem generalizes by stating that $A$ is unitarily equivalent to a multiplication operator.

This is: $\exists$ an isometry $U$ and a measure space such that 
\begin{displaymath}
  (U^{-1}AUf)(x)=\lambda(x)f(x),
\end{displaymath}
where $\lambda(x)$ is the spectrum of $A$, and the operator acts like multiplication by $\lambda(x)$.
\end{frame}
%%%%%
\begin{frame}{The Position Operator}
Thus, by the spectral theorem; any state can be written as a linear combination of eigenvectors of the given operator.

Thus we can interpret 
\begin{displaymath}
  \psi(q) = \int_{-\infty}^{\infty}\delta(q-q^{\prime})\psi(q^{\prime})dq^{\prime}
\end{displaymath}
as the expansion of an arbitrary state in terms of a continuous linear combination of eigenvectors of $Q$ with eigenvalue $q^{\prime}$.
\end{frame}
%%%%%%%%%%%%%%%%%%%%%%%%%
\section[The Momentum Space Representation]{The Momentum Space Representation}
%%%%%
\begin{frame}{Momentum Representation}
Let $\mathcal{H}$ be a Hilbert space. Going back to the Hamiltonian of the free particle, by taking states to be wavefunctions $\psi(q)$, we can Fourier transform them, as
\begin{displaymath}
  \tilde\psi(k) = \mathcal{F}\psi = \frac{1}{\sqrt{2\pi}}\int_{-\infty}^{\infty}\exp(-ikq)\psi(q)dq.
\end{displaymath}
And even more, we can consider $\mathcal{H}$ to be a space of functions $\tilde\psi(k)$ on momentum space.
\end{frame}
%%%%%
\begin{frame}{Momentum Operator}
Given a Hilbert space $\mathcal{H}$ with the momentum operator
\begin{displaymath}
  P\tilde\psi(k)=k\tilde\psi(k),
\end{displaymath}
we will call them momentum space representation.

Note:By the Plancherel theorem, momentum and space representations are unitarily equivalent representations of the group $\mathbb{R}$.
\end{frame}
%%%%%
\begin{frame}{Eigenfunctions of P}
In this representation, the eigenfunctions of $P$ are distributions $\delta(k-k^{\prime})$, with eigenvalue $k^{\prime}$, and the expansion of any state reads
\begin{displaymath}
  \tilde\psi(k)=\int_{-\infty}^{\infty}\delta(k-k^{\prime})\tilde\psi(k^{\prime})dk^{\prime}.
\end{displaymath}

And even more, the position operator is
\begin{displaymath}
  Q=i\frac{d}{dk},
\end{displaymath}
with eigenfunctions 
\begin{displaymath}
  \frac{1}{\sqrt{2\pi}}\exp(-ikq{\prime})
\end{displaymath}
\end{frame}
%%%%%%%%%%%%%%%%%%%%%%%%%
\section[Dirac Notation]{Dirac Notation}
%%%%%
\begin{frame}{Dirac Notation}
In Dirac notation, $Q$ and $P$ are
\begin{displaymath}
  Q|q\rangle=q|q\rangle,   P|k\rangle=k|k\rangle,
\end{displaymath}
and arbitrary states
\begin{displaymath}
  \langle q|\psi\rangle = \psi(q), \langle k|\psi\rangle = \psi(k).
\end{displaymath}
Proper interpretation of relations 
\begin{displaymath}
  \langle q|q^{\prime}\rangle=\delta(q-q^{\prime}),\langle k|k^{\prime}\rangle=\delta(k-k^{\prime})
\end{displaymath}
\end{frame}
%%%%%
\begin{frame}{Dirac Notation}
We have 
\begin{displaymath}
  |\psi\rangle=\int_{-\infty}^{\infty}|q\rangle\langle q|\psi \rangle dq,
\end{displaymath}
\begin{displaymath}
  |\psi\rangle=\int_{-\infty}^{\infty}|k\rangle\langle k|\psi \rangle dk,
\end{displaymath}
with identity
\begin{displaymath}
  1 = \int_{-\infty}^{\infty}|q\rangle\langle q| dq=\int_{-\infty}^{\infty}|k\rangle\langle k| dk
\end{displaymath}
\end{frame}
%%%%%
\begin{frame}{Dirac Notation: Switching representations 1}
Transformation between both bases/representations is done by Fourier transform 
\begin{displaymath}
  \langle k|\psi\rangle=\int_{-\infty}^{\infty}\langle k|q\rangle \langle q|\psi\rangle dq
\end{displaymath}
and
\begin{displaymath}
  \langle k|q\rangle = \frac{1}{\sqrt{2\pi}}\exp(-ikq).
\end{displaymath}
\end{frame}
%%%%%
\begin{frame}{Dirac Notation: Switching representations 2}
Transformation between both bases/representations is done by Fourier transform 
\begin{displaymath}
  \langle q|\psi\rangle=\int_{-\infty}^{\infty}\langle q|k\rangle \langle k|\psi\rangle dk
\end{displaymath}
and
\begin{displaymath}
  \langle q|k\rangle = \frac{1}{\sqrt{2\pi}}\exp(ikq).
\end{displaymath}
\end{frame}
%%%%%
\begin{frame}{Interlude 3: Bra-Ket Notation and Dual Spaces}
\begin{itemize}
	\item A state vector $|\psi\rangle$ (ket) lives in $\mathcal{H}$
	\item Its dual $\langle \psi|$ (bra) lives in $\mathcal{H}^{*}$ (the dual space)
	\item When we write $\langle \psi|\phi\rangle$, we're actually applying a linear functional from $\mathcal{H}^{*}$ to the complex numbers.
	\item Every observable corresponds to a linear operator $A$ that acts on states. 
	\item When we measure an observable, we're essentially using elements of the dual space
\end{itemize}
\end{frame}
%%%%%%%%%%%%%%%%%%%%%%%%%
\section{Heisenberg uncertainty}
%%%%%
\begin{frame}{Theorem}
\begin{displaymath}
  \frac{\langle\psi|Q^2|\psi\rangle}{\langle\psi|\psi\rangle}\frac{\langle\psi|P^2|\psi\rangle}{\langle\psi|\psi\rangle}\geq\frac{1}{4}
\end{displaymath}
The proof relies on using self-adjointness of $P$ and $Q$ together with commutation relations between $Q$ and $P$.
\end{frame}
%%%%%%%%%%%%%%%%%%%%%%%%%
\section{The Propagator in Position Space}
%%%%%
\begin{frame}{Time Evolution}
For any quantum system, time evolution if given by the unitary operator 
\begin{displaymath}
  U(t) = \exp(-itH),
\end{displaymath}
considering that $H$ is independent of time.
\begin{itemize}
	\item Momentum Space: just the multiplication operator.
	\item Position Space: given by an integral kernel called the propagator.
\end{itemize}
\end{frame}
%%%%%
\begin{frame}{Interlude 4: Time Evolution}
If $H$ is time dependent but the commutes at different times, then the time evolution is given by 
\begin{displaymath}
U(t) = \exp[-\frac{1}{\hbar}\int_{0}^{t}dt^{\prime}H(t^{\prime})]
\end{displaymath}

If $H$ is time dependent and at different times do not commute, then the time evolution is given by
\begin{displaymath}
	U(t) = 1+\sum_{n=1}^{\infty} (\frac{-i}{\hbar})^n\int_{0}^{t_1}dt_{1}\cdots\int_{0}^{t_{n-1}}dt_n H(t_1)\cdots H(t_n)
\end{displaymath}
\end{frame}
%%%%%
\begin{frame}{Propagator}
Definition: The position space propagator is the kernel $U(t, q_t, q_0)$ of the time evolution operator acting on position space wavefunctions. Determines the time evolution of wavefunctions for all times t by
\begin{displaymath}
  \psi(q_t,t) = \int_{-\infty}^{\infty} U(t, q_t, q_0)\psi(q_0,0)dq_0,
\end{displaymath}
where $\psi(q_0,0)$ is the initial value of the wavefunction at time 0.
\end{frame}
%%%%%
\begin{frame}{Propagator: Dirac Notation}
Using Dirac notation we have
\begin{displaymath}
  \psi(q_t,t) = \langle q_t|\psi(t)\rangle = \langle q_t|\exp(-itH)|\psi(0)\rangle,
\end{displaymath}
\begin{displaymath}
  \implies \psi(q_t,t) = \langle q_t|\exp(-itH)\int_{-\infty}^{\infty}|q_0\rangle\langle q_0|\psi(0)\rangle dq_0,
\end{displaymath}
and the propgator can be written as 
\begin{displaymath}
  U(t,q_t,q_0) = \langle q_t|\exp(-itH)|q_0\rangle
\end{displaymath}
\end{frame}
%%%%%
\begin{frame}{Propagator for the Free Particle}
For the free particle we can compute the progator, which is 
\begin{displaymath}
  U(t,q_t,q_0) = \frac{1}{2\pi}\int_{-\infty}^{\infty}\exp(-ik(q_t - q_0))\exp(-\frac{ik^2 t}{2m})dk,
\end{displaymath}
and $U(t,q_t,q_0) = U(t,q_t-q_0)$.
due to translation invarianceo of $H$, the propagator only depends on the difference $q_t-q_0$.
\end{frame}
%%%%%
\begin{frame}{Tricks from Complex Analysis}
Let's consider $it\rightarrow z=\tau+it$, then the propagator is well defined when $\tau=Re(z)>0$, which defined a holomorphic function in $z$. Then we can find
\begin{displaymath}
  U(z,q_t-q_0)=\sqrt{\frac{m}{2\pi z}}\exp(-\frac{m}{2z}(q_t-q_0)^2)
\end{displaymath}
\end{frame}
%%%%%
\begin{frame}{Relation with Heat Equation (Diffusion Equation)}
If $z=\tau$ is real and positive, then this is the kernel function for solutions to the partial differential equation
\begin{displaymath}
  \frac{\partial}{\partial \tau}\psi(q,t) = \frac{1}{2m}\frac{\partial^2}{\partial q^2}\psi(q,\tau),
\end{displaymath}
which models the way temperature diffuses in a medium, it also models the way probability of a given position diffuses in a random walk.
\end{frame}
%%%%%
\begin{frame}{Interlude 5: Fokker-Planck Equation}
The FP equation is a classical PDE that describes the time evolution of a probability density function:
\begin{displaymath}
  \frac{\partial}{\partial t}p(x,t) = -\frac{\partial}{\partial x}(\mu(x,t)p(x,t)) + \frac{\partial^2}{\partial x^2}(D(x,t)p(x,t))
\end{displaymath}

\begin{itemize}
	\item Solving certain FPEs using quantum mechanical methods.
	\item The quantum extensions of the FPE help describe phenomena like decoherence, thermalization, and energy relaxation in quantum systems interacting with their environment.
\end{itemize}
\end{frame}
%%%%%%%%%%%%%%%%%%%%%%%%%
\section{Propagators in frequency-momentum space}
%%%%%
\begin{frame}{Causality}
The propagator as defined before works for positive and negative times, let's define a version that takes into account causality

Definition(Retarted Propagator): The retarted propagator $U_{+}(t,q_t-q_0)$ is given by $0$ if $t<0$ and $U_{+}(t,q_t-q_0)=U(t,q_t,q_0)$ if $t>0$.

It can also be wrtitten in terms of a step function $\theta$
\end{frame}
%%%%%
\begin{frame}{Integral Representation of $\theta$}
A useful representation of $\theta$ is given by 
\begin{displaymath}
  \theta(t) = \lim_{\epsilon\rightarrow 0^{+}}\frac{i}{2\pi}\int_{-\infty}^{\infty}\frac{1}{\omega + i\epsilon}\exp(-i\omega t)d\omega.
\end{displaymath}	
\end{frame}
%%%%%
\begin{frame}{Propagator in frequency domain}
The Fourier transform of the previously copmuted propagator is
\begin{displaymath}
  \hat U(\omega,k) = \frac{1}{\sqrt{2\pi}}\int_{-\infty}^{\infty}(\frac{1}{\sqrt{2\pi}}\exp(-\frac{ik^2 t}{2m})\exp(i\omega t)dt,
\end{displaymath}
$\implies$ 
\begin{displaymath}
	\hat U(\omega,k) = \delta(\omega - \frac{1}{2m}k^2)  
\end{displaymath}
\end{frame}
%%%%%
\begin{frame}{Retarded Propagator in position space}
The retarded propagator in position space is given by 
\begin{displaymath}
  U_{+}(t,q_t - q_0) = \lim_{\epsilon\rightarrow 0^{+}}(\frac{1}{2\pi})^{2}\int_{-\infty}^{\infty}\int_{-\infty}^{\infty}\frac{i}{\omega + i\epsilon}f(\omega, k,t,q_t-q_0)d\omega dk,
\end{displaymath}
where 
\begin{displaymath}
f(\omega, k,t,q_t-q_0) = \exp(-i(\omega + \frac{1}{2m}k^2)t)\exp(ik(q_t-q_0))
\end{displaymath}
\end{frame}
%%%%%
\begin{frame}{Retarded Propagator in position space}
Doing the change of variables $\omega\rightarrow \omega^{\prime} =\omega + \frac{1}{2m}k^2$, we can find
\begin{displaymath}
  U_{+}(t,q_t - q_0) = \lim_{\epsilon\rightarrow 0^{+}}(\frac{1}{2\pi})^{2}\int_{-\infty}^{\infty}\int_{-\infty}^{\infty}g d\omega^{\prime} dk
\end{displaymath}
where \begin{displaymath}
  g=\frac{exp(-i\omega^{\prime} t)\exp(ik(q_t-q_0))}{\omega^{\prime}-\frac{1}{2m}k^2+i\epsilon}
\end{displaymath}
\end{frame}
%%%%%
\begin{frame}{{Retarded Propagator in position space}}
Which is the Fourier transform 
\begin{displaymath}
  U_{+}(t,q_t - q_0) =\frac{1}{\sqrt{2\pi}}\int_{-\infty}^{\infty}\int_{-\infty}^{\infty} \hat U_{+}(\omega, k) hd\omega dk,
\end{displaymath}
where
\begin{displaymath}
  \hat U_{+}(\omega, k) = \lim_{\epsilon\rightarrow 0^{+}}\frac{i}{2\pi}\frac{1}{\omega -\frac{1}{2m}k^2+i\epsilon}
\end{displaymath}
and
\begin{displaymath}
	h = \exp(-i\omega t)\exp(ik(q_t-q_0))
\end{displaymath}
\end{frame}
%%%%%%%%%%%%%%%%%%%%%%%%%
\section{Green's Functions and Solutions to the Schrodinger Equation	}
%%%%%
\begin{frame}{Green's Function}
\begin{itemize}
	\item Act as an integral kernel that transforms differential equations into more manageable algebraic forms.
	\item In position basis, the Green's function acts as a propagator, describing how quantum particles move between different positions
\end{itemize}
Given the PDE
\begin{displaymath}
  D\psi=J,
\end{displaymath}
We define the Green's function of $D$ to be the distribution with Fourier transform
\begin{displaymath}
  \hat G = \frac{1}{\hat D}
\end{displaymath}
\end{frame}
%%%%%
\begin{frame}{Schrodinger Equation}
Let $D$ be given by 
\begin{displaymath}
  D = i\frac{\partial}{\partial t}+\frac{1}{2m}\frac{\partial^2}{\partial q^2},
\end{displaymath}
then, it's Fourier transform will be
\begin{displaymath}
  \hat D = \omega - \frac{k^2}{2m},
\end{displaymath}
ans thus, it's Green function will be
\begin{displaymath}
  \hat G = \frac{1}{\omega - \frac{k^2}{2m}}
\end{displaymath}
\end{frame}
%%%%%
\begin{frame}{Schrodinger Equation}
And then, solutions of $D\psi=J$, are found by computing the inverse Fourier transform of $\hat G\hat J$
\begin{displaymath}
  \psi(q,t) = \frac{1}{2\pi}\int_{-\infty}^{\infty}\int_{-\infty}^{\infty}\frac{1}{\omega -\frac{k^2}{2m}}\hat J(\omega, k)\exp(-i\omega t)\exp(ikq)d\omega dk
\end{displaymath}
\end{frame}
%%%%%
\begin{frame}{Final Words: $P$ and $Q$ Representations}
Complementary Descriptions of Quantum Systems linked by Fourier transformations;
\begin{itemize}
	\item $Q$ representation describes the system using a wavefunction, which gives the probability amplitude for finding the particle at a specific position.
	\item $P$ representation describes the system using a wavefunction, that provides the probability amplitude for the particle having a specific momentum.
	\item In $Q$ representation $Q$ is multiplicative and $P$ is a differential operator.
	\item in $P$ representation $P$ is multiplicative and $Q$ is a differential operator.
\end{itemize}
\end{frame}
%%%%%
%%%%%
%%%%%
%%%%%
%%%%%
\end{document}
