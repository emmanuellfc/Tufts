\documentclass[oneside,english]{amsart}
\usepackage{charter}
\usepackage[T1]{fontenc}
\usepackage[latin9]{inputenc}
\usepackage{geometry}
\geometry{verbose,tmargin=4cm,bmargin=4cm,lmargin=3cm,rmargin=3cm}
\usepackage{fancyhdr}
\pagestyle{fancy}
\setlength{\parskip}{\smallskipamount}
\setlength{\parindent}{0pt}
\usepackage{amsthm}
\usepackage{amssymb}
\usepackage{setspace}
\onehalfspacing

\makeatletter
\numberwithin{equation}{section}
\numberwithin{figure}{section}
\newtheorem{Proof}{proof}
\makeatother

\usepackage{babel}
\begin{document}
\title{Classwork 8}
\date{\today}
\author{Danshyl Boodhoo, Isaac Dreeben, Anna J. Kelley, Chris Kotsifakis, Emmanuel Flores}

\maketitle
Let $(X, \mathcal{T})$ be a topological space. Show that the set of connected components of $X$ forms a partition of X.
\begin{proof}
The definition of connected components is given by
\begin{itemize}
	\item A connected component is a maximal connected subset.
	\item Each point belong to exactly one connected component.
\end{itemize}
Now, to show, that the connected components of $X$ form a partition, we need to show that they are pairwise disjoint and their union is the whole $X$. 
Pairwise disjoint: Let $C_{1}$ and $C_{2}$ be two connected components of $X$, and by contradiction let's assume they have a commont point $p$, now let's look at $C = C_1 \cup C_2$, which is a connected subset of $X$, but whis contradicts the maximality of both $C_{1}$ and $C_{2}$, it follows that distinct connected components must be disjoint.
Union is all $X$: every point in $X$ belongs to at least one connected component, otherwise it would contradict the definition of connected components.

$\therefore$ the set of connected components forms a partition of X.
\end{proof}




\end{document}
