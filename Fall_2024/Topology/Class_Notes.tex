\documentclass[12pt]{article}
% Formatting Page and Font Type
\usepackage{geometry}
\geometry{letterpaper, portrait, margin=1in}
\usepackage{charter}
% Math packages and environments
\usepackage{amsthm}
\usepackage{amssymb}
\newtheorem{definition}{Definition}
\newtheorem{theorem}{Theorem}
\newtheorem{corollary}{Corollary}[theorem]
\newtheorem{lemma}[theorem]{Lemma}
% Title
\title{Topology: Class Notes}
\author{Emmanuel Flores \thanks{eq.emmanuel@gmail.com}}
\date{\today}


\begin{document}
\maketitle

\begin{section}{General Defintions}

\end{section}





\begin{section}{Bases and Subspaces}

In what follows I'm going to use TS for Topological Space. Given a topological space we define a basis for this TS as follows:

\begin{definition}[Basis]
Given a TS $\left( X, \mathcal{T}\right)$, we say that $\mathcal{\beta}\subset\mathcal{T}$ is a basis for the TS if all open subsets of $X$ can be written as the union of elements of $\mathcal{\beta}$
\end{definition}

\begin{theorem}
Let $\left( X, \mathcal{T}\right)$ be a TS, and let $\mathcal{\beta}$ be a collection of subsets of $X$. Then $\mathcal{\beta}$ is a basis for $\mathcal{T}$ if and only if
\begin{itemize}
	\item $\mathcal{\beta}\subset\mathcal{T}$,
	\item for each open set $U$ in $\mathcal{T}$ and a point $p\ni U$ there is a set $V\in\mathcal{\beta}$ such that $p\in V\subset U$

\begin{proof}
Let's assume that $\mathcal{\beta}$ is a basis, thus it follows that $\mathcal{\beta}\subset\mathcal{T}$ by definition. On the other hand, let $U\in\mathcal{T}$ such that $p\in U$, because $\mathcal{\beta}$ is basis it follows that $U$ can be written as the union of elements of $\mathcal{\beta}$, and from this it follows that there exists $V\in\mathcal{\beta}$ such that$p\in V\subset U$.

Now let's $\mathcal{\beta}$ as given 
\end{proof}	
	
\end{itemize}
\end{theorem}
\end{section}

\end{document}