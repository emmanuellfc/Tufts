\documentclass[12pt]{article}
%%%%%%%%%%%%%%%%%%%%%%%%%%%%%%%%%%% Formatting Page and Font Type
\usepackage{charter}
\usepackage{geometry}
\geometry{letterpaper, portrait, margin=1in}
\usepackage{tcolorbox}
%%%%%%%%%%%%%%%%%%%%%%%%%%%%%%%%%%% Math packages and environments
\usepackage{amsthm}
\usepackage{amssymb}
\newtheorem{definition}{Definition}[section]
\newtheorem{proposition}{Proposition}[section]
\newtheorem{theorem}{Theorem}[section]
\newtheorem{corollary}{Corollary}[section]
\newtheorem{lemma}{Lemma}[section]
%%%%%%%%%%%%%%%%%%%%%%%%%%%%%%%%%%% Title
\title{Topology: Class Notes}
\author{Emmanuel Flores \thanks{eq.emmanuel@gmail.com}}
\date{\today}

%%%%%%%%%%%%%%%%%%%%%%%%%%%%%%%%%%%
\begin{document}
\maketitle
%%%%%%%%%%%%%%%%%%%%%%%%%%%%%%%%%%%
%%%%% General Definitions
\begin{section}{General Definitions}

Before we define the idea of topological space I think it's worth saying a few words about it, in particular I think it's quite important to say that the first books on the subject were under the title of set theory, which explains why the heavy use of that "apparatus" in the theory. Ok, having said that Let's start with the definition of topological space.

\begin{tcolorbox}[title=Topological Space]
	\begin{definition}[Topological Space]
	 A topological space is a pair $(X,\mathcal{T})$ consisting of a a non-empty set $X$, and a family $\mathcal{T}\subset \mathcal{P}(X)$, satisfying the following properties \begin{itemize}
  			\item $\emptyset\in \mathcal{T},X\in \mathcal{T}$,
  			\item $\mathcal{T}$ is closed under arbitrary unions,
  			\item $\mathcal{T}$ is closed under finite intersections.
		\end{itemize}

	\end{definition}
\end{tcolorbox}
\end{section}

%%%%%%%%%%%%%%%%%%%%%%%%%%%%%%%%%%%
%%%%% Bases, SubBases and Subspaces

\begin{section}{Bases and Subspaces}

As far as I know the idea of basis for a topological space is pretty similar to the idea of a basis in linear algebra. The goal is to describe the whole space in terms of its building blocks. And in the context of topology the building blocks are open sets, however in linear algebra the idea was to use the basic elements with linear combinations in order to generate the whole space. In this case we don't have the concept of linear space, but the concept of union and intersection of sets; therefore, we should expect that the definition of basis follows one of those operations.
On the other hand in the definition of topology we learn that the arbitrary union of open sets is also and open set, which in some sense implies that the result of those union could be quite complicated; this is where the idea of basis becomes useful.
In what follows I'm going to use TS for Topological Space. Given a topological space we define a basis for this TS as follows:
% TODO Add definitions about closure, interior and boundary. 
\begin{definition}[Basis]
Given a TS $\left( X, \mathcal{T}\right)$, we say that $\mathcal{\beta}\subset\mathcal{T}$ is a basis for the TS if all open subsets of $X$ can be written as the union of elements of $\mathcal{\beta}$
\end{definition}

\begin{theorem}
Let $\left( X, \mathcal{T}\right)$ be a TS, and let $\mathcal{\beta}$ be a collection of subsets of $X$. Then $\mathcal{\beta}$ is a basis for $\mathcal{T}$ if and only if
	\begin{itemize}
		\item $\mathcal{\beta}\subset\mathcal{T}$,
		\item for each open set $U$ in $\mathcal{T}$ and a point $p\ni U$ there is a set $V\in\mathcal{\beta}$ such that $p\in V\subset U$
	\end{itemize}
\end{theorem}

\begin{tcolorbox}
By definition of basis it follows that the empty set is the union of open sets, in this case an empty union is the empty set, but what does that even mean?
\end{tcolorbox}

\begin{theorem}
Suppose that $X$ is a TS and $\mathcal{B}$ is a collection of subsets of $X$. Then $\mathcal{B}$ is a basis for some topology on $X$ if and only if 
	\begin{itemize}
		\item each point of $X$ is in some element of $\mathcal{B}$, and
		\item if $U$ and $V$ are sets in $\mathcal{B}$ and $p\in U\cap V$, there is a set $W$ in $\mathcal{B}$ such that $p\in W\subset U\cap V$
	\end{itemize}
\end{theorem}

\begin{definition}
Suppose $\mathcal{T}$ and $\mathcal{T}^{\prime}$ are two topologies on the same underlying set $X$. If $\mathcal{T}\subset\mathcal{T}^{\prime}$, then we say that $\mathcal{T}^{\prime}$ is finer than $\mathcal{T}$. Alternatively, we say that $\mathcal{T}$ is coarser than  $\mathcal{T}^{\prime}$. We say strictly coarser or strictly finer if additionally $\mathcal{T}\neq\mathcal{T}^{\prime}$
\end{definition}

\begin{subsection}{Subbases}

\begin{definition}
Let $\left(X, \mathcal{T}\right)$ a TS and let $\mathcal{S}$ be a collection of subsets of $X$. Then $\mathcal{S}$ is said to be a sub basis for $\mathcal{T}$ if and only if the collection $\mathcal{B}$ of all finite intersections of sets in $\mathcal{S}$ is a basis for $\mathcal{T}$. And an element of $\mathcal{S}$ is called a sub basis element or a subbasic open set.
\end{definition}

\begin{theorem}
Let $\left(X, \mathcal{T}\right)$ a TS and let $\mathcal{S}$ be a collection of subsets of $X$. Then $S$ is sub-basis for $\mathcal{T}$ if and only if
	\begin{enumerate}
	\item $\mathcal{S}\subset\mathcal{T},$ and
	\item for each set $U\in \mathcal{T}$ and a point $p\in U$ there is an finite collection $\{ V_{i}\}_{I}^{n}$ of elements of $\mathcal{S}$ such that 
	$$ 
		p\in \cup_{j=1}^{n}V_{j}\subset U
	$$
	\end{enumerate}
\end{theorem}

\begin{theorem}
Suppose $X$ is a set and $\mathcal{S}$ is a collection of subsets of $X$. Then $\mathcal{S}$ is a sub basis for some topology on $X$ if and only if every point of $X$ is in some element of $\mathcal{S}$.
\end{theorem}

\end{subsection}

\begin{subsection}{Order Topology}

\begin{definition}
Let $X$ be a totally ordered set by $\leq$. Let $\mathcal{B}$ be the collection of all subsets of $X$ that are any of the following terms
$$
	\{ x\in X | x < a \}, \{ x\in X | a < x\}, \{x\in X | a< x < b\}.
$$
For $a,b\in X$. Then $\mathcal{B}$ is a basis for a topology $\mathcal{T}$ called the order topology on $X$.
\end{definition}

\begin{definition}
Given sets $A$ and $B$, their product ( or Cartesian Product) $A\times B$ is the set of all ordered pairs $(a,b)$ such that $a\in A$ and $b\in B$.
\end{definition}

\end{subsection}

\begin{subsection}{Subspaces}
As in the case of basis, we can sort of make a connection with concepts from linear algebra; in that area of mathematics we have the notion of subspace, and also pretty much in al other areas of mathematics, but let's stick with this particular one. So having defied a topology in a given set $X$ we can take a subset of this space, let's call it $Y$ and form a topology on this is space, formally we have the following definition.

\begin{definition}
Let $\left(X, \mathcal{T}\right)$ a TS. For $Y\subset X$, the collection 
$$
	\mathcal{T}_{Y}=\{ U| U = V \cap Y, V\in \mathcal{T}\}
$$
is a topology on $Y$ called the subspace topology. It's also called the relative topology on $Y$ inherited from $X$. And the space $\left( Y,\mathcal{T}_{Y}\right)$ is called a subspace of $X$. And even more, if $O\in\mathcal{T}_{Y}$ we say that $U$ is open in $Y$.
\end{definition}

\begin{theorem}
Let $\left(X, \mathcal{T}\right)$ a TS. Then the collection of sets $\mathcal{T}_{Y}$ is in fact a topology on $Y$.
\end{theorem}

\begin{theorem}
Let $\left(Y, \tau_{Y} \right)$ be a subspace of $\left(X, \mathcal{T}\right)$. A subset $C\subset Y$ is closed in $\left(Y, \tau_{Y} \right)$ if and only if there is a set $D\subset X$, closed in $\left(X, \mathcal{T}\right)$ such that $C=D\cap Y$
\end{theorem}

\begin{corollary}
Let $\left(Y, \tau_{Y} \right)$ be a subspace of $\left(X, \mathcal{T}\right)$. A subset $C\subset Y$ is closed in $\left(Y, \tau_{Y} \right)$ if and only if $Cl_{X}(C)\cap Y = C$
\end{corollary}

And now, let's proceed with the connection between basis in a space and subspaces.

\begin{theorem}
Let $\left(X, \mathcal{T}\right)$ a TS, and let $\left(Y, \tau_{Y} \right)$ be a subspace. If $\mathcal{B}$ is a basis for $\mathcal{T}$, then $\mathcal{B}_{Y} = \{B\cap Y | B \in \mathcal{B} \}$ is a basis for $\mathcal{T}_{Y}$
\end{theorem}
\end{subsection}

\begin{subsection}{Product Spaces}

\begin{definition}
Let $X$ and $Y$ be two sets. The projection functions $\pi_{X}:X\times Y\rightarrow X$ and $\pi_{X}:X\times Y\rightarrow Y$ are defined by $\pi_{X}(x,y) = x$ and $\pi_{Y}(x,y) = y$
\end{definition}

\begin{definition}
Suppose that $X$ and $Y$ are topological spaces. The product topology on the product $X\times Y$ is the topology whose basis is all the elements of the form $U\times V$, where U is an open set in $X$ and $V$ is an open set in $Y$.
\end{definition}

\begin{theorem}
Show that the product topology on $X\times Y$ is the same as the topology generated by the sub basis of inverse images of open sets under the projection functions, that is, the sub basis is $\{ \pi_{X}^{-1}(U) | U\in\mathcal{T}_{X} \}\cup \{ \pi_{X}^{-1}(V) | V\in\mathcal{T}_{Y} \}$
\end{theorem}

\begin{definition}
Let $\{ X_{\alpha}\}_{\alpha\in\lambda}$ be a collection of topological spaces. The product $\Pi_{\alpha\in\lambda}X_{\alpha}$, or Cartesian product, is the set of functions 
$$
\{ f:\lambda\rightarrow\cup_{\alpha\in\lambda}X_{\alpha} | \forall \alpha\in\lambda, f(\alpha)\in X_{\alpha} \}.
$$
And in this defintion $f(\alpha)$ is the $\alpha$-th coordinate of $f$, whereas the spaces $X_{\alpha}$ are sometimes called factors of the infinite product.
\end{definition}

\end{subsection}
\end{section}
%%%%%%%%%%%%%%%%%%%%%%%%%%%%%%%%%%%
%%%%% Covers

\begin{section}{Open Cover and Compactness}
\begin{definition}
	$(X, \mathcal{T})$ is said to be compact if and only if, for every open cover of $X$, we can find a finite subcover.
\end{definition}

\begin{proposition}
	Suppose that $X$ is a compact TS, the the following statements hold:
	\begin{enumerate}
		\item Every sequence in $X$ has at least one limit point.
		\item If a sequence has a unique limit point, then it converges to this point.
	\end{enumerate}
\end{proposition}

\begin{theorem}
	Suppose $(X, T)$ is a compact TS. Then it follows that 
	\begin{enumerate}
		\item Every infinite subset $A\subset X$ has at least one accumulation point.
		\item Every subset $A$ of $X$ which has no accumulation point is finite.
	\end{enumerate}
\end{theorem}

\begin{proposition}
	Let $A$ be a Hausdorff subspace of a topological space $X$. $A$ is compact if and only if every family of open sets in $X$ which covers $A$ contains a finite family which covers $A$.
\end{proposition}

\begin{theorem}
	Let $(X, T)$ be a Hausdorff space, If $A$ is a compact subspace then $A$ is closed in $X$.
\end{theorem}

\begin{theorem}
	Let $X$ and $Y$ be two compact topological spaces. Then $E = X\times Y$ with the product topology is also compact.
\end{theorem}

\begin{theorem}
	Let $X, Y$ be topological spaces and $f:X\rightarrow Y$, then the following statements are true:
	\begin{enumerate}
		\item $f$ is continuous.
		\item $f^{-1}(C)$ is closed for each $C\subset Y$ closed.
		\item For every set  $A\subset X$ we have $f(\overline{A})\subset\overline{f(A)}$
	\end{enumerate}
\end{theorem}

\begin{proposition}
	Let $E,F,G$ be TS together with $f:E\rightarrow F$ and $g:F\rightarrow G$ both continuous functions, thus the composition $g\circ f:E\rightarrow G$ is also continuous.
\end{proposition}

\begin{definition}
	Let $E,F$ be TS. A function $f:E\rightarrow F$ is said to be closed provided that $f(C)$ is closed in $F$ for any closed set $C$ in $E$.
	On the other hand, $f$ is said to be open provided that $f(O)$ is open for each open set $O$ in $E$.
\end{definition}


\begin{definition}
	Let $E, F$ be TS. A function $f:E\rightarrow F$ is said to be homeomorphic provided that $f$ is bijective, continuous, and $f^{-1}$ is also continuous. If $f:E\rightarrow F$ a homeomorphism, we say that $E$ and $F$ are homeomorphic.
\end{definition}

\begin{theorem}
	If $f:E\rightarrow F$ is continous, then the following statements are equivalent:
	\begin{enumerate}
		\item $f$ is homeomorphic.
		\item $f$ is a closed bijection.
		\item $f$ is an open bijection.
	\end{enumerate}
\end{theorem}

\begin{definition}
	Let $E\times F$ be the Cartesian product of two sets. The projection map $\pi_E$ and $\pi_F$ are defined as follows:
	\begin{displaymath}
  		\pi_E:E\times F\rightarrow E, \pi_F:E\times F\rightarrow F,
	\end{displaymath}
\end{definition}

\begin{proposition}
	If $E$ and $F$ are TS, and $E\times F$ is equipped with the product topology, then the projection maps are continuous.
\end{proposition}

\begin{theorem}
	Let $E$ and $F$ be TS. The product topology on $E\times F$ is the weakest topology that makes the projection maps $\pi_E$ and $\pi_F$ continuous on $E\times F$.
\end{theorem}

\begin{proposition}
	If $E$ and $F$ are TS and $f\in F$, then $E\times{f}$ is homeomorphic to $E$.
\end{proposition}

\begin{proposition}
	Let $E, F, G$ be TS. A function $f:G\rightarrow E\times F$ is continous if and only if $\pi_E\circ f$ and $\pi_F\circ f$ are continuous.
\end{proposition}



\end{section}

%%%%%%%%%%%%%%%%%%%%%%%%%%%%%%%%%%%
%%%%% Separation Properties

\begin{section}{Separation Properties}
One of the main motivations for topology is to understand the essential properties of a space that make ideas from calculus work, such as convergence and continuity. And in particular, we want to define those ideas without recurring to the concept of distance.

\begin{subsection}{Hausdorff, Regular and Normal Spaces}
\begin{definition}
Let $\left(X, \mathcal{T}\right)$ a TS.
	\begin{enumerate}
		\item $X$ is a $T_{1}$-space if and only if for every pair $x,y$ of distinct points there are open sets $U,V$ such that $x\in U$, $y\in V$ but $x\notin V$ and $y\notin U$.
		\item $X$ is Hausdorff, or a $T_{2}$-space, if and only if for every pair of distinct points $x,y$ there are disjoint open sets $U, V$ such that $x\in U$, $y\in V$.
	\end{enumerate}
\end{definition}
\end{subsection}





\end{section}
%%%%%%%%%%%%%%%%%%%%%%%%%%%%%%%%%%%%%%%%%%%%%%%%%%
%%%%%%%%%%%%%%%%%%%%%%%%%%%%%%%%%%%%%%%%%%%%%%%%%%
\end{document}