%% LyX 2.3.7 created this file.  For more info, see http://www.lyx.org/.
%% Do not edit unless you really know what you are doing.
\documentclass[oneside,english]{amsart}
\usepackage{mathpazo}
\usepackage[T1]{fontenc}
\usepackage[latin9]{inputenc}
\usepackage{geometry}
\geometry{verbose,tmargin=4cm,bmargin=4cm,lmargin=3cm,rmargin=3cm}
\usepackage{fancyhdr}
\pagestyle{fancy}
\setlength{\parskip}{\smallskipamount}
\setlength{\parindent}{0pt}
\usepackage{amsthm}
\usepackage{amssymb}
\usepackage{setspace}
\onehalfspacing

\makeatletter
%%%%%%%%%%%%%%%%%%%%%%%%%%%%%% Textclass specific LaTeX commands.
\numberwithin{equation}{section}
\numberwithin{figure}{section}

\makeatother

\usepackage{babel}
\begin{document}
\title{Classwork: Week of Sep 16, 2024}
\author{Xavier S. deSouza, Cisco J. Hadden, Seth R. Lupo, Abhi Mummaneni,
Emmanuel Flores}

\maketitle
Let $X$ be a nonempty set and $\mathcal{T}$ be the set of all subset
$E$ of $T$ such that $X/E$ is countable or $E=\emptyset$.
\begin{enumerate}
\item Prove that $\left(X,\mathcal{T}\right)$ is a topological space. ($\mathcal{T}$
is called the countable complement topology on $X$.)

1. Let's prove that $\emptyset,X\in\mathcal{T}$. Indeed, $\emptyset\in\mathcal{T}$
by definition. On the other hand, $X\in\mathcal{T}$ because $X/X=\emptyset$
which is a member of $\mathcal{T}$.

2. Let's prove that the finite intersection of open sets is an open
set. Indeed, let $U_{1},U_{2}\in\mathcal{T}$, then, by the definition
of $\mathcal{T}$, we have that $U_{1}$ is empty or $X/U_{1}$ countable,
and the same as well for $X/U_{2}$. So let's assume that both $U_{1}$
and $U_{2}$ are non-empty, thus, it follows that
\[
X/U_{1}\hspace{1em}\&\hspace{1em}X/U_{2},
\]
are both countable, now let's take $U_{1}\cap U_{2}$, we want to
prove that $U_{1}\cap U_{2}\in\mathcal{T}$. Indeed, if $U_{1}\cap U_{2}=\emptyset$,
then $U_{1}\cap U_{2}\in\mathcal{T}$, now, let's assume that the
intersection is non-empty, thus using the Morgan's Laws, we have that
\[
X/\left(U_{1}\cap U_{2}\right)=\left(X/U_{1}\right)\cup\left(X/U_{2}\right),
\]
but we know that the union of countable sets is countable, and by
assumption $X/U_{1}$ and $X/U_{2}$ are countable, thus we have that
\[
U_{1}\cap U_{2}\in\mathcal{T},
\]
just as we wanted.

3. Now, we want to prove that the arbitrary union of open sets is
open. Indeed let $U_{\alpha}$ be an indexed family of open sets,
with $\alpha\in\lambda$. Then, let's consider $\cup_{\alpha\in\lambda}U_{\alpha}$,
and again, we have two options, for each $\alpha\in\lambda$ either
$U_{\alpha}$ is the empty set or the complement is countable, so
let's assume that the complement is countable. Again, using the Morgan's
Laws, we have that 
\[
X/\left(\cup_{\alpha\in\lambda}U_{\alpha}\right)=\cap_{\alpha\in\lambda}\left(X/U_{\alpha}\right),
\]
but each one of $X/U_{\alpha}$ are countable, thus the intersection
is at most countable, with means that 
\[
\cup_{\alpha\in\lambda}U_{\alpha}\in\mathcal{T},
\]
just as we wanted.
\item For $X=\mathbb{R}$ given an example of a set that is open in both
the standard and countable complement topologies.

The standard topology of $\mathbb{R}$ is given by the open intervals,
thus if we consider 
\[
\mathcal{U}=\left(-\infty,0\right)\cup\left(0,\infty\right),
\]
then, using the definition of countable topology we have that 
\[
\mathbb{R}/\mathcal{U}=\left\{ 0\right\} ,
\]
which is finite, and therefore, countable.
\item For $X=\mathbb{R}$ given an example of a set that is open in the
standard topology but not open in the countable complement topology.

Again, the standard topology of $\mathbb{R}$ is given by the open
intervals, thus if we consider the open interval $\left(0,1\right)$
and taking complement, we have 
\[
X/\left(0,1\right)=\left(-\infty,0\right]\cup\left[1,\infty\right),
\]
which is clearly non countable, thus we found an open set which is
open in the standard topology but not in the countable complement
topology.
\item For $X=\mathbb{R}$ given an example of a set that is closed in both
the standard and countable complement topologies.

The complement of a closed set is an open set, thus $0$ is an closed
set in the standard topology because the complement is the union of
two open sets, this is 
\[
X/\left\{ 0\right\} =\left(-\infty,0\right)\cup\left(0,\infty\right),
\]
on the other hand in the countable complement topology the complements
of open sets are the empty set of countable, thus $\left\{ 0\right\} $
is closed too in the countable complement topology.
\end{enumerate}

\end{document}
