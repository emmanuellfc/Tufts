\documentclass[11pt]{article}
\usepackage{geometry}
\geometry{letterpaper, portrait, margin=1in}
\usepackage{amsmath}
\usepackage{amssymb}
\usepackage{breqn}
\usepackage{charter}
\usepackage{amsthm}
\usepackage{thmtools}
\declaretheoremstyle[headfont=\normalfont]{normalhead}
\declaretheorem[style=normalhead]{problem}
\declaretheorem[style=normalhead]{solution}
%%%%%%%%%%%%%%%%%%%%%%%%%%%%%%%%%%%%%%%%%%%%%%%%%%%%%%%%%%%%
\title{Structure Formation, Statistics, and Scalar Field}
\author{J. Emmanuel Flores}
%%%%%%%%%%%%%%%%%%%%%%%%%%%%%%%%%%%%%%%%%%%%%%%%%%%%%%%%%%%%
\begin{document}
\maketitle
%%%%%%%%%%%%%%%%%%%%%%%%%%%%%%%%%%%%%%%%%%%%%%%%%%%%%%%%%%%%
\begin{problem}
Dark Matter and Baryon Density Growth.
\newline
\textbf{a.} Let's start by defining 
\begin{displaymath}
  \epsilon = \delta_b(t) - \delta_d(t),
\end{displaymath}
and we're given
\begin{displaymath}
  \ddot\delta_d(t) + 2H\dot\delta_d(t) - \frac{3}{2}H^2(\Omega_d\delta_d +\Omega_b\delta_b)=0,
\end{displaymath}
\begin{displaymath}
  \ddot\delta_b(t) + 2H\dot\delta_b(t) - \frac{3}{2}H^2(\Omega_d\delta_d +\Omega_b\delta_b)=0,
\end{displaymath}
by taking the difference of the two previous equations, we have the following
\begin{displaymath}
  \ddot\delta_b(t)-\ddot\delta_d(t) + 2H\dot\delta_b(t) -2H\dot\delta_d(t)=0\implies \ddot\delta_b(t)-\ddot\delta_d(t) + 2H(\dot\delta_b(t) -\dot\delta_d(t))=0
\end{displaymath}
but we know that $H=12/3t$, thus
\begin{displaymath}
\ddot{\epsilon} + \frac{4}{3t}\dot{\epsilon}=0,
\end{displaymath}
\newline
\textbf{b.} Let's seek solutions of the previous equation with the following form
\begin{displaymath}
  \epsilon =  t^n,
\end{displaymath}
then, we have
\begin{displaymath}
 t^{n-2}\left(n(n-1)+\frac{4}{3}n\right)=0\iff n\left(n+\frac{1}{3}\right) = 0
\end{displaymath}
which implies that
\begin{displaymath}
 n\in\{0,-\frac{1}{3}\},
\end{displaymath}
and with this, the general solution will be
\begin{displaymath}
  \epsilon(t) = \epsilon_0 + \frac{\epsilon_1}{t^{1/3}},
\end{displaymath}
and from the form of the solution, we can see that a late times $\epsilon\rightarrow \epsilon_0$, which implies that at late times
\begin{displaymath}
 \delta_b-\delta_d\rightarrow\epsilon_0\implies \frac{\delta_b-\delta_d}{\delta_b}\rightarrow\frac{\epsilon_0}{\delta_b}\implies 1-\frac{\delta_d}{\delta_b}\rightarrow\frac{\epsilon_0}{\delta_b},
\end{displaymath}
which is valid for all $\epsilon_0$, and in particular is valid for $\epsilon_0=0$, thus we have
\begin{displaymath}
  \frac{\delta_d}{\delta_b}\rightarrow 1.
\end{displaymath}
\newline
\textbf{c.} I append the Mathematica notebook with the solution and the corresponding plots.
\newline
\textbf{d.} From the numerical plot we can see that a late times $\delta_{b}$ becomes almost equal to $\delta_{d}$ which is in agreement with that I found in b. And finally, 
\begin{displaymath}
  t_{dec} = 3.17432\times10^7\text{ years}.
\end{displaymath}

\end{problem}
\newpage
%%%%%%%%%%%%%%%%%%%%%%%%%%%%%%%%%%%%%%%%%%%%%%%%%%%%%%%%%%%%
\begin{problem}
Matter Growth with Dark Energy
\newline
\textbf{a.} Let's take derivatives
\begin{displaymath}
  \frac{d\delta}{dt} = \frac{da}{dt}\frac{d\delta}{da}\implies \dot\delta=\delta^\prime \dot a,
\end{displaymath}
where $^\prime$ means derivative with respect to $a$. On the other hand, we also have
\begin{displaymath}
  \ddot \delta = \frac{d}{dt}(\dot\delta) = \frac{d}{dt}(\delta^\prime \dot a) = \dot a\frac{d\delta^\prime}{dt} + \delta^\prime\frac{d}{dt}(\dot a) \implies \ddot\delta =  \dot a^2\delta^{\prime\prime} + \delta^\prime\ddot a,
\end{displaymath}
just as we wanted.
\newline
\textbf{b.} The first Friedmann equation with just matter and dark matter is given by
\begin{displaymath}
  \frac{H^2}{H_0^2} = \Omega_{M0} a^{-3} + \Omega_{V0}
\end{displaymath}
and from this we have
\begin{displaymath}
  \frac{\dot a^2}{a^2} = H_0^2\left(\Omega_{M0} a^{-3} + \Omega_{V0} \right),
\end{displaymath}
which implies that 
\begin{displaymath}
\dot a^2= a^2 H_0^2\left(\Omega_{M0} a^{-3} + \Omega_{V0} \right).
\end{displaymath}
Now, if we take the time derivative of the previous expression we have
\begin{displaymath}
2\dot a\ddot a = 2a\dot aH_0^2\left(\Omega_{M0} a^{-3} + \Omega_{V0} \right) + a^2H_0^2(-3\Omega_{M0} a^{-2} \dot a),
\end{displaymath}
which implies that
\begin{displaymath}
2\dot a\ddot a = 2a\dot aH_0^2\left(\Omega_{M0} a^{-3} + \Omega_{V0} \right) - 3\dot a a^{-2}H_0^2\Omega_{M0},
\end{displaymath}
and by cancelind the $\dot a$ we have
\begin{displaymath}
2\ddot a = 2aH_0^2\left(\Omega_{M0} a^{-3} + \Omega_{V0} \right) - 3 a^{-2}H_0^2\Omega_{M0}, 
\end{displaymath}
and from this we have
\begin{displaymath}
2\ddot a = -a^{-2}H_0^2\Omega_{M0} + 2aH_0^2\Omega_{V0},
\end{displaymath}
and by factorizing some terms 
\begin{displaymath}
 2\ddot a = -aH_0^2(\Omega_{M0}a^{-3} - 2\Omega_{V0}),
\end{displaymath}
which leads us to
\begin{displaymath}
\ddot a = -aH_0^2(\Omega_{M0}a^{-3} - 2\Omega_{V0})/2
\end{displaymath}
just as we wanted.
\newline
\textbf{c.} Starting with 
\begin{displaymath}
  \ddot\delta + 2H\dot\delta - \frac{3}{2}H^2\Omega_{M0}\delta=0,
\end{displaymath}
and using the previous results we have that
\begin{displaymath}
a^2H_0^2\left(\Omega_{M0} a^{-3} + \Omega_{V0}\right) \delta^{\prime\prime} + \delta^{\prime}(-aH_0^2(\Omega_{M0}a^{-3} - 2\Omega_{V0})/2) + 2H\delta^{\prime}\dot{a} -\frac{3}{2}H^2\Omega_{M0}\delta=0
\end{displaymath}
\newline
\textbf{d.} Let's look at both limit cases
\begin{itemize}
\item By considering $a$ small and dark matter dominating, we have that $\Omega_{M0}/a^3$ dominates, and therefore, the equation becomes
\begin{displaymath}
  a^2(\Omega_{M0}/a^3)\delta^{\prime\prime}+\frac{3}{2}(\Omega_{M0}/a^3)\delta^{\prime} -\frac{3}{2}\Omega_{M0}/a^3\delta=0,
\end{displaymath}
which implies that
\begin{displaymath}
  a^2\delta^{\prime\prime} + \frac{3}{2}\delta^\prime-\frac{3}{2}\delta=0,
\end{displaymath}
and if we seek solutions of the form $\delta=a^n$, we'll have the following equation
\begin{displaymath}
  a^n(n^2+\frac{1}{2}n-\frac{3}{2})=0,
\end{displaymath}
with solutions $n=-3/2$ and $n=1$, therefore $\delta\propto a$ is a valis solution.
\item On the other hand $a$ is large and dark energy dominates, we have that $\Omega_{M0}/a^3\ll 1$, which implies that terms without $\Omega_{M0}/a^3\ll 1$ will dominate, and in limit the equation will take the form
\begin{displaymath}
 a^2(1-\Omega_{M0})\delta^{\prime\prime} + 3a(1-\Omega_{M0})\delta^{\prime}=0,
\end{displaymath}
and from here we can see that $\delta$ proportional to a constant is a valid solution.
\end{itemize}
Therefore such possible scenarios have either $\delta\propto a$ or $\delta\propto$ a constant.
\newline
\textbf{e.} I append a Mathematica notebook with the graph.
\newline
\textbf{f.} In the plot I made I include a fit taken into account the firts points, wich indeed resemble a linear function in $a$, whereas for higer values of $a$ one can see that the solution reached a plateau, which is consistent witht the previous analysis in both scenarios/limits.
\end{problem}
\newpage
%%%%%%%%%%%%%%%%%%%%%%%%%%%%%%%%%%%%%%%%%%%%%%%%%%%%%%%%%%%%
\begin{problem}
Power Spectrum
\newline
By definition $\langle\delta(\mathbf{k})\delta^{*}(\mathbf{k})^\prime\rangle$ is given by
\begin{displaymath}
\langle\delta(\mathbf{k})\delta^{*}(\mathbf{k})^\prime\rangle = \int d^3 xd^3 x^{\prime} \exp(-i\mathbf{k}\cdot \mathbf{x})\exp(-i\mathbf{k}\cdot \mathbf{x}^\prime)\langle\delta(\mathbf{x})\delta(\mathbf{x^\prime})\rangle,
\end{displaymath}
and if we make $\mathbf{r} = |\mathbf{x}-\mathbf{x}^\prime|$ we have
\begin{displaymath}
\exp(-i\mathbf{k}\cdot\mathbf{x}+i\mathbf{k}^\prime\cdot\mathbf{x}^\prime) = \exp(-i\mathbf{k}\cdot\mathbf{x} + i\mathbf{k}\cdot\mathbf{x}^\prime -i\mathbf{k}\cdot\mathbf{x}^\prime + i\mathbf{k}^\prime\cdot\mathbf{x}^\prime),
\end{displaymath}
then we have
\begin{displaymath}
\exp(-i\mathbf{k}\cdot\mathbf{x}+i\mathbf{k}^\prime\cdot\mathbf{x}^\prime) = \exp[-i\mathbf{k}\cdot(\mathbf{x}-\mathbf{x}^\prime)-i\mathbf{x}\cdot(\mathbf{k}-\mathbf{k}^\prime)],  
\end{displaymath}
\begin{displaymath}
\implies \exp(-i\mathbf{k}\cdot\mathbf{x}+i\mathbf{k}^\prime\cdot\mathbf{x}^\prime) = \exp(-i\mathbf{k}\cdot\mathbf{r})\exp(-i\mathbf{x}\cdot(\mathbf{k}-\mathbf{k}^\prime)). 
\end{displaymath}
With this information, we can arrive at the following expression
\begin{displaymath}
\langle\delta(\mathbf{k})\delta^{*}(\mathbf{k})^\prime\rangle = \int d^3 rd^3 x^\prime e^{-i\mathbf{k}\cdot\mathbf{r}}e^{-i\mathbf{x}\cdot(\mathbf{k}-\mathbf{k}^\prime)}\xi(r),
\end{displaymath}
where
\begin{displaymath}
  \xi(r) = \langle\delta(\mathbf{x})\delta(\mathbf{x^\prime})\rangle.
\end{displaymath}
And if now we integrate over the $x^\prime$ variables we have
\begin{displaymath}
\langle\delta(\mathbf{k})\delta^{*}(\mathbf{k})^\prime\rangle = (2\pi)^3\delta^{3}(\mathbf{x}-\mathbf{x}^\prime)\int d^3 r e^{-i\mathbf{k}\cdot\mathbf{r}}\xi(r),
\end{displaymath}
and by making
\begin{displaymath}
  \mathcal{P}(k) = \int d^3 r e^{-i\mathbf{k}\cdot\mathbf{r}}\xi(r),
\end{displaymath}
we arrive at the following result
\begin{displaymath}
\langle\delta(\mathbf{k})\delta^{*}(\mathbf{k})^\prime\rangle = (2\pi)^3\delta^{3}(\mathbf{x}-\mathbf{x}^\prime)\mathcal{P}(k)
\end{displaymath}
just as we wanted.
\end{problem}
\newpage
%%%%%%%%%%%%%%%%%%%%%%%%%%%%%%%%%%%%%%%%%%%%%%%%%%%%%%%%%%%%
\begin{problem}
Scalar Field (Inflaton) in Expanding Universe
\newline
\textbf{a.} Let's begin with the following action
\begin{displaymath}
  S = \int d^4x \sqrt{-g}\left[\frac{1}{2} g^{\mu\nu}\partial_\mu\phi\partial_\nu\phi -V(\phi)\right],
\end{displaymath}
where 
\begin{displaymath}
  g_{\mu\nu}=\text{diag}(1,-a^2,-a^2,-a^2),
\end{displaymath}
and with this metric, we have
\begin{displaymath}
\sqrt{-g}=a^3,
\end{displaymath}
whereas 
\begin{displaymath}
  g^{\mu\nu} = \text{diag}(1,-1/a^2,-1/a^2,-1/a^2),
\end{displaymath}
therefore, the action takes the form
\begin{displaymath}
S = \int dt d^3x a^3\left[\frac{1}{2}(\dot\phi)^2 - \frac{1}{2a^2}(\nabla\phi)^2-V(\phi) \right]
\end{displaymath}
which can be simplified to
\begin{displaymath}
S = \int dt d^3x \left[\frac{1}{2}a^3(\dot\phi)^2 - \frac{1}{2}a(\nabla\phi)^2- a^3V(\phi) \right].
\end{displaymath}
And if we perform the variation, we have
\begin{displaymath}
  \delta S = \int dt d^3x \left[a^3\dot\phi\delta\dot\phi - a\nabla\phi\cdot\nabla\delta\phi - a^3 \frac{V(\phi)}{d\phi}\delta\phi \right],
\end{displaymath}
for the first two terms, we can perform integration by parts as follows
\begin{displaymath}
\int dt(a^3 \dot\phi)\delta\dot\phi = -\int\frac{d}{dt}\left(a^3\dot\phi\right)\delta\phi + \text{boundary terms},
\end{displaymath}
and 
\begin{displaymath}
\int d^3x a\nabla\phi\cdot\nabla\delta\phi = -\int d^3x (a\nabla^2\phi)\delta\phi + \text{boundary terms},
\end{displaymath}
thus, if we neglect the boundary terms, the variation becomes
\begin{displaymath}
  \delta S = \int dt d^3x\left[-\frac{d}{dt}\left(a^3\dot\phi\right)\delta\phi + (a\nabla^2\phi)\delta\phi - a^3 \frac{V(\phi)}{d\phi}\delta\phi\right],
\end{displaymath}
and impossing the condition $\delta S=0$ we have
\begin{displaymath}
  -\frac{d}{dt}\left(a^3\dot\phi\right)\delta\phi + (a\nabla^2\phi)\delta\phi - a^3 \frac{V(\phi)}{d\phi}\delta\phi = 0,
\end{displaymath}
which can be simplified to 
\begin{displaymath}
  -3a^2\dot a\dot\phi  - a^3\ddot\phi + a\nabla^2\phi - a^3 \frac{V(\phi)}{d\phi}=0,
\end{displaymath}
since $a\neq 0$ we have
\begin{displaymath}
 -3\frac{\dot a}{a}\dot\phi  - \ddot\phi + \frac{\nabla^2\phi}{a^2} -\frac{V(\phi)}{d\phi}=0,
\end{displaymath}
and by using the fact $H = \dot a /a$ we have
\begin{displaymath}
-3H\dot\phi  - \ddot\phi + \frac{\nabla^2\phi}{a^2} -\frac{V(\phi)}{d\phi}=0, 
\end{displaymath}
which can also be writen as
\begin{displaymath}
\ddot\phi + 3H\dot\phi- \frac{\nabla^2\phi}{a^2} + \frac{V(\phi)}{d\phi}=0,
\end{displaymath}
just as we wanted.
\newline
\textbf{b.} By assuming $\phi$ is homogeneous in space and that $V=\frac{1}{2}m^2\phi^2$ we have
\begin{displaymath}
  \frac{dV}{d\phi} = m^2\phi, \nabla^2\phi = 0,
\end{displaymath}
thus the previous equation becomes:
\begin{displaymath}
\ddot\phi + 3H\dot\phi + m^2\phi=0,  
\end{displaymath}
just as we wanted.
\newline
\textbf{c.} Finally, by assuming $H$ a constant we seek solutions of the form 
\begin{displaymath}
  \phi = \phi_0\exp(rt),
\end{displaymath}
we have the following equation
\begin{displaymath}
\phi_0\exp(rt)\left(r^2 +3Hr+m^2\right)=0
\end{displaymath}
and solving the characteristic equation lead us to the following expression
\begin{displaymath}
  r = \frac{-3H\pm \sqrt{1-\frac{2m^2}{9H^2}}}{2},
\end{displaymath}
which can be simplified to 
\begin{displaymath}
r = \frac{3H}{2}\left( 1 \pm \sqrt{1-\left(\frac{2m}{3H}\right )^2}\right)
\end{displaymath}
or we can also make
\begin{displaymath}
  r = \frac{m}{\alpha}(1 \pm \sqrt{1-\alpha^2}),
\end{displaymath}
where $\alpha = \frac{2m}{3H}$. Therefore, the general solution takes the form
\begin{displaymath}
  \phi(t) = \phi_{+}\exp\left[\frac{m}{\alpha}(1 + \sqrt{1-\alpha^2})t\right] + \phi_{-}\exp\left[\frac{m}{\alpha}(1 - \sqrt{1-\alpha^2})t\right].
\end{displaymath}
Now, the solution will oscillate when the following condition hold
\begin{displaymath}
  1-\left(\frac{2m}{3H}\right)^2<0 \iff 2m>3H.
\end{displaymath}
On the other hand, the solution will not oscillate when
\begin{displaymath}
  1-\left(\frac{2m}{3H}\right)^2 \geq 0 \iff 3H\geq 2m.
\end{displaymath}
And finally, if $m\ll H$, this implies that $\frac{m}{H}\ll 1$, thus the solution will take the form
\begin{displaymath}
  r\approx 3H\implies \phi(t) \propto e^{3Ht}
\end{displaymath}
therefore, $\phi$ will evolve quickly.
\end{problem}
\end{document}