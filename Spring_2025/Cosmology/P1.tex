\documentclass[11pt]{article}
\usepackage{geometry}
\usepackage{amsmath}
\usepackage{amssymb}
\usepackage{breqn}
\usepackage{charter}
\usepackage{amsthm}

\usepackage{geometry}
\geometry{letterpaper, portrait, margin=1in}

\usepackage{thmtools}
\declaretheoremstyle[headfont=\normalfont]{normalhead}
\declaretheorem[style=normalhead]{problem}
\declaretheorem[style=normalhead]{solution}

%%%%%%%%%%%%%%%%%%%%%%%%%%%%%%%%%%%%%%%%%%%%%%%%%%
\title{Problem 1: Evolution of Universe with general Energy of Particles}
\author{Emmanuel Flores}
\date{\today}
%%%%%%%%%%%%%%%%%%%%%%%%%%%%%%%%%%%%%%%%%%%%%%%%%%
\begin{document}
\maketitle
The Friedmann equation is given by 
\begin{displaymath}
  \left( \frac{\dot{R}}{R}\right)^2 = \frac{8\pi G}{3}\rho +\frac{2E}{mR^2},
\end{displaymath}
where \begin{displaymath}
  \rho = \frac{\rho_0 R_0^3}{R^3}.
\end{displaymath}
%%%%%%%%%%%%%%%% Problem 1
\begin{problem}
	Rewrite the Friedmann equation by replacing derivatives with respect to the time variable $t$ by derivatives with respect to the conformal time variable $\eta$ defined by
\end{problem} 
\begin{displaymath}
  dt = Rd\eta
\end{displaymath}

%%%%%%%%%%%%%%%% Solution 1
\begin{solution}
By the chain rule, we have
\begin{displaymath}
  \frac{dR}{dt} = \frac{dR}{d\eta}\frac{d\eta}{dt}\implies \frac{dR}{dt} = \frac{dR}{d\eta}\frac{1}{R},
\end{displaymath}
and from this, the Friedmann equation is rewriten as
\begin{displaymath}
  \left( \frac{dR}{d\eta}\frac{1}{R^2}\right)^2 = \frac{8\pi G}{3}\rho +\frac{2E}{mR^2},
\end{displaymath}
thus
\begin{displaymath}
  \left( \frac{dR}{d\eta}\right)^2 = \frac{8\pi GR^4}{3}\rho +\frac{2ER^4}{mR^2},
\end{displaymath}
but $\rho$ is a function of $R$, thus
\begin{displaymath}
  \left( \frac{dR}{d\eta}\right)^2 = \frac{8\pi \rho_0 R_0^3G}{3}R +\frac{2E}{m}R^2.
\end{displaymath}
and by making
\begin{displaymath}
  \alpha = \frac{8\pi \rho_0 R_0^3G}{3}, \beta = \frac{2E}{m},
\end{displaymath}
we have
\begin{displaymath}
  \left(\frac{dR}{d\eta}\right )^2 = \alpha R+\beta R^2.
\end{displaymath}
\end{solution}

\pagebreak
%%%%%%%%%%%%%%%% Problem 2
\begin{problem}
Use separation of variables to re-write the Friedmann equation as an expression for $\eta$ in terms of an integral of $R$.
\end{problem}

%%%%%%%%%%%%%%%% Solution 2
\begin{solution}
	By separation of variables we have
\begin{displaymath}
  \frac{dR}{\sqrt{\alpha R+\beta R^2}} = d\eta,
\end{displaymath}
thus, the problem now resides on solving the preovious integral.
	
\end{solution}

\pagebreak
%%%%%%%%%%%%%%%% Problem 3
\begin{problem}
	Carry out the integral and invert to obtain $R(\eta)$; simplify for $E > 0$, $E = 0$, and $E < 0$.
\end{problem}

%%%%%%%%%%%%%%%% Solution 3
\begin{solution}
In the variables that I'm using $E = 0\implies \beta=0$, $E>0\implies \beta>0$, and $E<0\implies \beta<0$, So let's proceed accordingly.

\textbf{Case 1}, $\beta = 0$: in this case, the integral becomes 
\begin{displaymath}
  \int \frac{dR}{\sqrt{\alpha R }} = \int d\eta,
\end{displaymath}
with solution given by
\begin{displaymath}
  R(\eta) = \frac{\alpha}{2}\eta^2
\end{displaymath}

\textbf{Case 2}, $\beta < 0 $: Here, I'm going to add the sign of $\beta$ by hand (not elegant at all, but just to be more explicit),
\begin{displaymath}
    \int \frac{dR}{\sqrt{\alpha R-\beta R^2}} = \int d\eta,
\end{displaymath}
and here the idea is to make a change of variables (I attach the notes with the algebra at the end of the document). Having done that, the solution reads
\begin{displaymath}
  R(\eta) = \frac{\alpha}{\beta} \cos^2\left( \frac{\sqrt{\beta}}{2}\eta \right)
\end{displaymath}

\textbf{Case 3}, $\beta > 0 $:  Here the integral becomes
\begin{displaymath}
  \int \frac{dR}{\sqrt{\alpha R+\beta R^2}} = \int d\eta,
\end{displaymath}
with solution given by
\begin{displaymath}
  R(\eta) = \frac{\alpha e^{\alpha\eta}}{1-\beta e^{\alpha\eta}},
\end{displaymath}
and again, details about the algebra are attached at the end of the document.

\end{solution}

\pagebreak
%%%%%%%%%%%%%%%% Problem 4
\begin{problem}
	Determine $t(\eta)$; again simplify for $E > 0$, $E = 0$, and $E < 0$.
\end{problem}

%%%%%%%%%%%%%%%% Solution 4
\begin{solution}
We know that
\begin{displaymath}
    dt = Rd\eta,
\end{displaymath}
and in the previous bullet we found exactly $R(\eta)$, thus we only need to integrate, this is
\begin{displaymath}
  t(\eta) = \int d\eta R(\eta).
\end{displaymath}


\textbf{Case 1}, $\beta=0$: in this case we have
\begin{displaymath}
  t(\eta) = \frac{\alpha}{2} \int d\eta \left(\eta^2\right),
\end{displaymath}
with solution given by
\begin{displaymath}
  t = \frac{\alpha}{6}\eta^3.
\end{displaymath}


\textbf{Case 2}, $\beta<0$: here we have
\begin{displaymath}
  t(\eta) = \frac{\alpha}{\beta}\int d\eta \cos^2\left( \frac{\sqrt{\beta}}{2}\eta \right),
\end{displaymath}
with solution
\begin{displaymath}
  t = \frac{\alpha}{\beta}\frac{2\sin\left (\frac{\sqrt{\beta}}{2}\eta\right )}{\sqrt{\beta}}
\end{displaymath}


\textbf{Case 3}, $\beta>0$: and finally, we have
\begin{displaymath}
  t(\eta) = \alpha \int d\eta \frac{e^{\alpha\eta}}{1-\beta e^{\alpha\eta}}
\end{displaymath}
with solution given by
\begin{displaymath}
  t(\eta) = -\frac{\log \left(1-\beta  e^{\alpha  x}\right)}{\beta }.
\end{displaymath}
	
\end{solution}

\pagebreak
%%%%%%%%%%%%%%%% Problem
\begin{problem}
	Using the above pair of parametric equations, plot $R(t)$ (use some plotting software, such as Mathematica) for 3 different choices of $E$, namely $E > 0$, $E = 0$, and $E < 0$.
\end{problem}

%%%%%%%%%%%%%%%% Solution
\begin{solution}
By combining the two previous parts we can write a parametric equation
\begin{displaymath}
  \gamma(\eta) = (R(\eta), t(\eta)),
\end{displaymath}
therefore.

\textbf{Case 1}, $\beta=0$: in this case we have
\begin{displaymath}
  \gamma(\eta) = \frac{\alpha}{2}\left( \eta^2, \eta^3\right)
\end{displaymath}


\textbf{Case 2}, $\beta<0$: again, we have
\begin{displaymath}
  \gamma(\eta) = \frac{\alpha}{\beta}\left( \cos^2\left( \frac{\sqrt{\beta}}{2}t\right), \frac{2}{\sqrt{\beta}}\sin\left(\frac{\sqrt{\beta}}{2}t \right) \right)
\end{displaymath}

\textbf{Case 3}, $\beta>0$: and finally
\begin{displaymath}
  \gamma(\eta) = \left(\frac{\alpha e^{\alpha\eta}}{1-\beta e^{\alpha\eta}}, -\frac{\log \left(1-\beta  e^{\alpha  x}\right)}{\beta }\right)
\end{displaymath}
\end{solution}
The plots are shown in the next page.

%%%%%%%%%%%%%%%%%%%%%%%%%%%%%%%%%%%%%%%%%%%%%%%%%%
\end{document}