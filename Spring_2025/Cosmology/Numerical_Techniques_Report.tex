\documentclass[twocolumn, 10pt, letterpaper]{article}
\usepackage{geometry}
\geometry{letterpaper, portrait, margin=0.75in}
\usepackage{amsmath}
\usepackage{amssymb}
\usepackage{graphicx}
\usepackage{hyperref}
\usepackage{breqn}
\usepackage{amsthm}
\usepackage{thmtools}
%%%%%%%%%%%%%%%%%%%%%%%%%%%%%%%%%%%%%%%%%%%%%%%%%%%%%%%%%%%%
\title{Numerical Methods in Cosmology}
\author{Emmanuel Flores}
\date{\today}
%%%%%%%%%%%%%%%%%%%%%%%%%%%%%%%%%%%%%%%%%%%%%%%%%%%%%%%%%%%%
\begin{document}
\maketitle
\abstract{The follwing paper provides a quick overview of some of the standar numerical techniques used in the field of cosmolgy and it's }
\section{Motivation and Background}
It's clear that there are some systems that are quite difficult to handle from an analytical perspective. And since is difficult that does not mean we should't do it, but instead that paralled to the progress along those lines, we should explore complementary paths to gain insight and understaning of the problem at hand.
\subsection{GR Essentials}
Here I will plan to describe the notation and main equations
\section{A quick Overview of Current Numerical Techniques in Cosmology}
\subsection{Background Observables}
\subsection{Non Linear Perturbation Theory}
\section{Background Dynamics and Machine Learning}
\subsection{Universal Approximation Theorem}
\subsection{Models under study}
\subsection{Methodology Details}
\subsection{Comparisson with other Approaches}
\section{Conclusions}

\end{document}
