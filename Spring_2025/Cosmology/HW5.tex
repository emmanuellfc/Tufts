\documentclass[11pt]{article}
\usepackage{geometry}
\geometry{letterpaper, portrait, margin=1in}
\usepackage{amsmath}
\usepackage{amssymb}
\usepackage{breqn}
\usepackage{charter}
\usepackage{amsthm}
\usepackage{thmtools}
\declaretheoremstyle[headfont=\normalfont]{normalhead}
\declaretheorem[style=normalhead]{problem}
\declaretheorem[style=normalhead]{solution}

\title{Cosmology, Problem Set 5: Neutron Abundance and Hydrogen Recombination}
\author{J. Emmanuel Flores}

\begin{document}
\maketitle
\textbf{Problem 1a:}
The Boltzmann equation reads 
\begin{displaymath}
  \frac{1}{a^3}\frac{d}{dt}(n_1 a^3) =-\Gamma_1 \left( n_1 -\left(\frac{n_1 n_2}{n_3 n_4}\right)_{eq} \frac{n_3 n_4}{n_2}\right),
\end{displaymath}
and by making $n_1 = n_n$, $n_3 = n_p$, with $n_2, n_4$ to be leptons $n_l$, we have
\begin{displaymath}
  \frac{1}{a^3}\frac{d}{dt}(n_n a^3) =-\Gamma_n \left( n_n -\left(\frac{n_n n_l}{n_p n_l}\right)_{eq} \frac{n_p n_l}{n_l}\right),
\end{displaymath}
which will lead us to
\begin{displaymath}
  \frac{1}{a^3}\frac{d}{dt}(n_n a^3) = -\Gamma_n \left( n_n -\left(\frac{n_n}{n_p}\right)_{eq} n_p\right),
\end{displaymath}
just as we wanted.
\newline
\textbf{Problem 1b:}
The equilibrium distribution of number density of species $i$ is given by 
\begin{displaymath}
  (n_i)_{eq} = g_{i}\left(\frac{m_i T}{2\pi}\right)^{3/2} \exp{\left[-\frac{m_i-\mu_i}{T}\right]},
\end{displaymath}
then by neglecting the chemical potential, we will have 
\begin{displaymath}
  (n_n)_{eq} = g_{n}\left(\frac{m_n T}{2\pi}\right)^{3/2} \exp{\left[-\frac{m_n}{T}\right]},
\end{displaymath}
and also
\begin{displaymath}
  (n_p)_{eq} = g_{p}\left(\frac{m_p T}{2\pi}\right)^{3/2} \exp{\left[-\frac{m_p}{T}\right]},
\end{displaymath}
from this we can make the ratio
\begin{displaymath}
  \left(\frac{n_i}{n_p}\right)_{eq} = \frac{g_{n}\left(\frac{m_n T}{2\pi}\right)^{3/2} \exp{\left[-\frac{m_n}{T}\right]}}{g_{p}\left(\frac{m_p T}{2\pi}\right)^{3/2} \exp{\left[-\frac{m_p}{T}\right]}}
\end{displaymath}
but $g_n = g_p$, thus
\begin{displaymath}
  \left(\frac{n_i}{n_p}\right)_{eq} = \left(\frac{m_n}{m_p}\right)^{3/2}\exp{\left[\frac{-m_n + m_p}{T}\right]},
\end{displaymath}
and if we define $Q = m_n-m_p$, we have $-Q = -m_n+m_p$, thus
\begin{displaymath}
  \left(\frac{n_i}{n_p}\right)_{eq} = \left(\frac{m_n}{m_p}\right)^{3/2}\exp{\left[\frac{-Q}{T}\right]},
\end{displaymath}
just as we wanted.
\newline
\textbf{Problem 1c:}
We know that 
\begin{displaymath}
  \frac{1}{a^3}\frac{d}{dt}(n_n a^3) = -\Gamma_n \left( n_n - \left(\frac{n_n}{n_p}\right)_{eq} n_p\right),
\end{displaymath}
on the other hand, if we define 
\begin{displaymath}
  X_n = \frac{n_n}{n_n + n_p},
\end{displaymath}
we can write
\begin{displaymath}
  \frac{1}{a^3}\frac{d}{dt}(n_n a^3 \frac{n_n + n_p}{n_n + n_p}) = -\Gamma_n \left( n_n - \left(\frac{n_n}{n_p}\right)_{eq} n_p\right),
\end{displaymath}
which implies that
\begin{displaymath}
  \frac{1}{a^3}\frac{d}{dt}(X_n a^3 (n_n + n_p)) = -\Gamma_n \left( n_n - \left(\frac{n_n}{n_p}\right)_{eq} n_p\right),  
\end{displaymath}
and if we take the time derivative we have
\begin{displaymath}
  \frac{1}{a^3}\left[a^3 (n_n + n_p)\frac{d}{dt}(X_n) + X_n\frac{d}{dt}(a^3 (n_n + n_p))\right] = -\Gamma_n \left( n_n - \left(\frac{n_n}{n_p}\right)_{eq} n_p\right).  
\end{displaymath}
On the other hand, if we assume that the baryon number is conserved, we have the following condition
\begin{displaymath}
  \frac{d}{dt}(a^3 (n_n + n_p))=0,
\end{displaymath}
which implies that
\begin{displaymath}
  \frac{1}{a^3}\left[a^3 (n_n + n_p)\frac{d}{dt}(X_n) \right] = -\Gamma_n \left( n_n - \left(\frac{n_n}{n_p}\right)_{eq} n_p\right),  
\end{displaymath}
thus
\begin{displaymath}
\frac{d}{dt} X_n = -\Gamma_n \left( \frac{n_n}{(n_n + n_p)} - \left(\frac{n_n}{n_p}\right)_{eq} \frac{n_p}{(n_n + n_p)}\right),  
\end{displaymath}
on the other hand we know that 
\begin{displaymath}
  \left(\frac{n_n}{n_p}\right)_{eq} \approx \exp(-Q/T),
\end{displaymath}
thus, we have
\begin{displaymath}
\frac{d}{dt} X_n = -\Gamma_n \left( X_n - \exp(-Q/T) \frac{n_p}{(n_n + n_p)}\right).
\end{displaymath}
On the other hand, we have the following identity
\begin{displaymath}
  \frac{n_p}{n_n +n_p} = 1 - \frac{n_n}{n_n +n_p}= 1 - X_n,
\end{displaymath}
which leads us to 
\begin{displaymath}
\frac{d}{dt} X_n = -\Gamma_n \left( X_n - \exp(-Q/T) (1 - X_n)\right),  
\end{displaymath}
therefore, we finally have
\begin{displaymath}
\frac{d}{dt} X_n = -\Gamma_n \left( X_n - (1 - X_n)\exp(-Q/T)\right),
\end{displaymath}
as desired.
\newline
\textbf{Problem 1d:}
For this part we have the following derivative
\begin{displaymath}
  \frac{dX_n}{dt}=\frac{dx}{dt}\frac{dX_n}{dx},
\end{displaymath}
but $x = x(T)$ and $T=T(a)$, thus by the chain rule we have
\begin{displaymath}
 	\frac{dX_n}{dt}=\frac{dx}{dT}\frac{dT}{da}\frac{da}{dt}\frac{dX_n}{dx},
\end{displaymath}
and more explicitly $x=Q/T$, whereas by assuming $T\propto a^{-1}$, we have
\begin{displaymath}
 	\frac{dx}{dT} = -\frac{Q}{T^2}, \frac{dT}{da}= -\frac{1}{a^2}, 
\end{displaymath}
 thus we have
\begin{displaymath}
	\frac{dX_n}{dt}=\left(-\frac{Q}{T^2}\right)\left(-\frac{1}{a^2}\right)\frac{da}{dt}\frac{dX_n}{dx},  
\end{displaymath}
\begin{displaymath}
\implies\frac{dX_n}{dt}=\left(\frac{Q}{T}\right)\left(\frac{1}{T}\right)\left(\frac{1}{a}\right)\left(\frac{1}{a}\frac{da}{dt}\right)\frac{dX_n}{dx},
\end{displaymath}
and again, using the fact that $T\propto a^{-1}$ we have $T/a = 1$, thus
\begin{displaymath}
\implies\frac{dX_n}{dt}=xH\frac{dX_n}{dx},  
\end{displaymath}
just as we wanted.
\newline
\textbf{Problem 1e:}
For this problem, I append the solution of the ode as a Mathematica notebook. On the other hand, the limi of neutron fraction for longer times is given by
\begin{displaymath}
  X_e \rightarrow 0.149533
\end{displaymath}

\newpage
\textbf{Problem 2a:}
 Starting with
\begin{displaymath}
  n_i = g_i\left(\frac{m_i T}{2\pi}\right)^{3/2} \exp\left[ -\frac{(m_i -\mu_i)}{T}\right],
\end{displaymath}
we have
\begin{displaymath}
  \frac{n_H}{n_e n_p} = \frac{g_H}{g_e g_p}\left( \frac{\left(\frac{m_H T}{2\pi}\right)^{3/2} }{\left(\frac{m_e T}{2\pi}\right)^{3/2} \left(\frac{m_p T}{2\pi}\right)^{3/2} }\right)\exp\left[ -\frac{(m_H -\mu_H)}{T} +\frac{(m_e -\mu_e)}{T}+\frac{(m_p -\mu_p)}{T} \right],
\end{displaymath}
and from this we have
\begin{displaymath}
  \frac{n_H}{n_e n_p} = \frac{g_H}{g_e g_p} \left(\frac{2\pi m_H}{m_e m_p T}\right)^{3/2} \exp\left[ \frac{m_e + m_p - m_H}{T} + \frac{\mu_H - \mu_e - \mu_p}{T}\right],
\end{displaymath}
and by using the condition of equilibrium given, we have 
\begin{displaymath}
  \mu_H - \mu_e - \mu_p =0,
\end{displaymath}
thus
\begin{displaymath}
  \frac{n_H}{n_e n_p} = \frac{g_H}{g_e g_p} \left(\frac{2\pi m_H}{m_e m_p T}\right)^{3/2} \exp\left[ \frac{m_e + m_p - m_H}{T}\right],
\end{displaymath}
and by making
\begin{displaymath}
  E_I = m_e + m_p - m_H,
\end{displaymath}
we finally have
\begin{displaymath}
  \frac{n_H}{n_e n_p} = \frac{g_H}{g_e g_p} \left(\frac{2\pi m_H}{m_e m_p T}\right)^{3/2} \exp\left[ \frac{E_I}{T}\right],
\end{displaymath}
just as we wanted.
\newline
\textbf{Problem 2b:}
The mass of the Hydrogen comes mostly from the proton, thus to a good approxixmation we have
\begin{displaymath}
  \frac{m_H}{m_p}\approx 1,
\end{displaymath}
thus this implies that
\begin{displaymath}
  \frac{n_H}{n_e n_p} = \frac{g_H}{g_e g_p} \left(\frac{2\pi }{m_e T}\right)^{3/2} \exp\left[ \frac{E_i}{T}\right],
\end{displaymath}
on the other hand, the degrees of freedom are $g_p=2$, $g_e=2$, whereas for the hydrogen, the spins of the electron and proton can be either aligned or anti aligned, which will give one singlet state and one triplet state, therefore $g_H = 1+3 =4$, and with this
\begin{displaymath}
  \frac{n_H}{n_e n_p} = \left(\frac{2\pi }{m_e T}\right)^{3/2} \exp\left[ \frac{E_I}{T}\right],
\end{displaymath}
moving on, the universe isn't electrically charged, so $n_e=e_p$, thus
\begin{displaymath}
  \frac{n_H}{n_e^2} = \left(\frac{2\pi }{m_e T}\right)^{3/2} \exp\left[\frac{E_I}{T}\right],
\end{displaymath}
as desired.
\newline
\textbf{Problem 2c:} If we define
\begin{displaymath}
  X_e = \frac{n_e}{n_e+n_H}, 
\end{displaymath}
then we have
\begin{displaymath}
  \frac{1-X_e}{X_e^2} = \frac{1-\frac{n_e}{n_e+n_H}}{\frac{n_e^2}{\left(n_e+n_H\right)^2}} = \frac{\frac{n_e+n_H-n_e}{n_e+n_H}}{\frac{n_e^2}{\left(n_e+n_H\right)^2}},
\end{displaymath}
thus, we can write
\begin{displaymath}
\frac{1-X_e}{X_e^2} = \frac{n_H}{n_e^2}\left( n_e + n_H\right) \implies   \frac{1-X_e}{X_e^2} = \frac{n_H}{n_e^2}n_b,
\end{displaymath}
where $n_b = n_e + n_H$.
\newline
\textbf{Problem 2d:} In the previous homework we found the following result for bosons
\begin{displaymath}
  n = \frac{\zeta(3)}{\pi^2}gT^3,
\end{displaymath}
and in particular, for photons we have $g=2$, thus
\begin{displaymath}
  n_\gamma = \frac{2\zeta(3)}{\pi^2}T^3,
\end{displaymath}
and if we use $n_b = \eta n_\gamma$ we have
\begin{displaymath}
  \frac{1-X_e}{X_e^2} = \frac{n_H}{n_e^2}\frac{2\eta\zeta(3)}{\pi^2}T^3,
\end{displaymath}
but we also know that
\begin{displaymath}
  \frac{n_H}{n_e^2} = \left(\frac{2\pi }{m_e T}\right)^{3/2} \exp\left[\frac{E_i}{T}\right],
\end{displaymath}
thus
\begin{displaymath}
  \frac{1-X_e}{X_e^2} = \left(\frac{2\pi }{m_e T}\right)^{3/2} \exp\left[\frac{E_i}{T}\right]\times\frac{2\eta\zeta(3)}{\pi^2}T^3,
\end{displaymath}
but $T^{-3/2}T^3 = T^{3/2}$, thus
\begin{displaymath}
 \frac{1-X_e}{X_e^2} = \left(\frac{2\pi T}{m_e}\right)^{3/2} \exp\left[\frac{E_i}{T}\right]\times\frac{2\eta\zeta(3)}{\pi^2},
\end{displaymath}
and we arrive at the desired result.
\newline
\textbf{Problem 2e:} I used Mathematica and I append the notebook. But here I'll do some algebra to find a nicer expression. 
The solution is given by
\begin{displaymath}
X_e(T)  = -\frac{e^{-\frac{E_I}{T}} \left(1-\sqrt{8 \sqrt{2} \pi
   ^{3/2} \eta  X_0 e^{E_I/T}
   \left(\frac{T}{m_e}\right){}^{3/2}+1}\right)}{4 \sqrt{2} \pi
   ^{3/2} \eta  X_0
   \left(\frac{T}{m_e}\right){}^{3/2}},
\end{displaymath}
where I defined 
\begin{displaymath}
  X_0 = \frac{2\zeta(3)}{\pi^2},
\end{displaymath}
but $8\times2^{1/2}=4\times2^{3/2}$, and $4\times 2^{1/2} = 2\times 2^{3/2}$, and with that we have
\begin{displaymath}
  X_e(T)  = -\frac{ \left(1 -\sqrt{4\times2^{3/2} \pi^{3/2} \eta  X_0 \exp[E_I/T] 
   \left(\frac{T}{m_e}^{}\right)^{3/2}+1}\right)}{2\times 2^{3/2} \pi
   ^{3/2} \eta  X_0 \left(\frac{T}{m_e}\right){}^{3/2}\exp[E_I/T]},
\end{displaymath}
which can be simplified to
\begin{displaymath}
  X_e(T) = \frac{-1+\sqrt{1 + 4 X_0\left(\frac{2\pi T}{m_e}\right)^{3/2} \exp\left[\frac{E_i}{T}\right]}}{2X_0\left(\frac{2\pi T}{m_e}\right)^{3/2} \exp\left[\frac{E_i}{T}\right]},
\end{displaymath}
and if we define 
\begin{displaymath}
  f = X_0\left(\frac{2\pi T}{m_e}\right)^{3/2} \exp\left[\frac{E_i}{T}\right],
\end{displaymath}
the final solution take the form
\begin{displaymath}
  X_e(T) = \frac{-1 + \sqrt{1+4f}}{2f},
\end{displaymath}
which is much "prettier".
\newline
\textbf{Problem 2f:} I append a Mathematica Notebook with the plot.
\newline
\textbf{Problem 2g:} The temperature of recombination can be calculated by
\begin{displaymath}
  X_e(T)=0.5,
\end{displaymath}
which I solved numerically, with the result given by
\begin{displaymath}
  T_{rec}=0.323887 \text{eV}.
\end{displaymath}
\newline
\textbf{Problem 2h:} For $a_{rec}$ we have
\begin{displaymath}
  T=\frac{T_0}{a}\implies a_{rec} = \frac{T_0}{T_{rec}},
\end{displaymath}
and if we use $T_0 = 2.3\times 10^{-4}$eV we have
\begin{displaymath}
  a_{rec} = 0.000710124.
\end{displaymath}
\newline
\textbf{Problem 2i:} Finally, we need to compute $t_{rec}$, where $t$ is given by
\begin{displaymath}
  t(a) = \frac{1}{H_0}\int_{0}^{a} \frac{dx}{x\sqrt{\Omega_{M,0}/x^3 + \Omega_{R,0}/x^4}},
\end{displaymath}
and if we use $a = a_{rec}$ together with $\Omega_{M,0} = 0.3$, $\Omega_{R,0}=8.6\times 10^-5$ and $H_0=0.7\times 10^{-10}$/years, we have
\begin{displaymath}
  t_{rec} = 243884 \text{ years}.
\end{displaymath}
\end{document}
