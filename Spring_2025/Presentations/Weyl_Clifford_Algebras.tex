%%%%%%%%%%%%%%%%%%%%%%%%%%%%%%%%%%%%
%%%%% Quantization
%%%%%%%%%%%%%%%%%%%%%%%%%%%%%%%%%%%%
%%%%%%%%%%%%%%%%%%%%%%%%%%%%%%%%%%%%
%%%%% Advanced Mathematical Methods
%%%%%%%%%%%%%%%%%%%%%%%%%%%%%%%%%%%%
\documentclass[12pt]{beamer}
\usepackage{amsmath}
\usetheme[progressbar=frametitle]{metropolis}
\title{Weyl and Clifford Algebras}
\author{Emmanuel Flores}
\institute{Advanced Mathematical Methods, \\Tufts University}

\begin{document}
\maketitle
%%%%%%%%%%%%%%%%%%%%%%%%%
\begin{frame}{Contents}
	\setbeamertemplate{section in toc}[sections numbered]
	\tableofcontents
\end{frame}
%%%%%%%%%%%%%%%%%%%%%%%%%
\section[Complex Weyl and Clifford Algebrass]{Complex Weyl and Clifford Algebrass}
\begin{frame}{One degree of freedon, bosonic case}
\begin{definition}
	The complex Weyl algebra in the one degree of freedom case is the algebra Weyl$(2, \mathbb{C})$, generated by the elements $1, a_{B}, a_{B}^{\dagger}$, satisfying the canonical commutation relations:
\begin{displaymath}
  [a_{B}, a_{B}^{\dagger}] = 1, [a_{B}, a_{B}] = [a_{B}^{\dagger}, a_{B}^{\dagger}]=0.
\end{displaymath}
\end{definition}
Weyl$(2, \mathbb{C})$is the algebra one gets by taking arbitrary products and complex linear combinations of the generators.
\end{frame}

\begin{frame}{One degree of freedon, bosonic case}
	
\end{frame}

\end{document}