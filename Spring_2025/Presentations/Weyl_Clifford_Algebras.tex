%%%%%%%%%%%%%%%%%%%%%%%%%%%%%%%%%%%%
%%%%% Quantization
%%%%%%%%%%%%%%%%%%%%%%%%%%%%%%%%%%%%
%%%%%%%%%%%%%%%%%%%%%%%%%%%%%%%%%%%%
%%%%% Advanced Mathematical Methods
%%%%%%%%%%%%%%%%%%%%%%%%%%%%%%%%%%%%
\documentclass[12pt]{beamer}
\usepackage{amsmath}
\usetheme[progressbar=frametitle]{metropolis}
\title{Weyl and Clifford Algebras}
\author{Emmanuel Flores}
\institute{Advanced Mathematical Methods, \\Tufts University}

\begin{document}
\maketitle
%%%%%%%%%%%%%%%%%%%%%%%%%
\begin{frame}{Contents}
	\setbeamertemplate{section in toc}[sections numbered]
	\tableofcontents
\end{frame}
%%%%%%%%%%%%%%%%%%%%%%%%%
\section[Complex Weyl and Clifford Algebrass]{Complex Weyl and Clifford Algebrass}
\begin{frame}{One degree of freedon, bosonic case}
\begin{definition}
	The complex Weyl algebra in the one degree of freedom case is the algebra Weyl$(2, \mathbf{C})$, generated by the elements $1, a_{B}, a_{B}^{\dagger}$, satisfying the canonical commutation relations:
\begin{displaymath}
  [a_{B}, a_{B}^{\dagger}] = 1, [a_{B}, a_{B}] = [a_{B}^{\dagger}, a_{B}^{\dagger}]=0.
\end{displaymath}

\end{definition}
\end{frame}

\begin{frame}{Procedure}
Given any classical system, we can quantize it by finding a rule as follows: for each function $f$ defined in the phase space, we associate a self-adjoint operator $\mathcal{O}_{f}$, acting on a state space $\mathcal{H}$, such that 
\begin{displaymath}
  O_{\{f,g\}} = -\frac{i}{\hbar}\left[ \mathcal{O}_f, \mathcal{O}_g\right]
\end{displaymath}
\end{frame}

\begin{frame}{Linear Functions and Schrodinger Representation}
"The Heisenberg Lie algebra is isomorphic to the threedimensional subalgebra of functions on phase space given by linear combinations of the constant function, the function q and the function p."
We have
\begin{displaymath}
  O_1 = \mathbf{1}, O_q = Q, O_p = P,
\end{displaymath}
with
\begin{displaymath}
  \Gamma'_{S}(1)=-i\mathbf{1},
\end{displaymath}
\begin{displaymath}
  \Gamma'_{S}(q) = -iQ = -iq,
\end{displaymath}
\begin{displaymath}
    \Gamma'_{S}(p) = -P=\frac{d}{dq}.
\end{displaymath}
\end{frame}

\begin{frame}{Quadratic Polynomials}
Quadratic polynomials can be quantized as follows
\begin{displaymath}
  O_{p^2/2} = \frac{P^2}{2}, O_{q^2/2} = \frac{Q^2}{2}
\end{displaymath}	
but, we need to work a more for $pq$ since the order here matters. It turns out that
\begin{displaymath}
  O_{pq} = \frac{1}{2}\left( PQ + QP\right).
\end{displaymath}
And the issue here is: "$\Gamma'_{S}$ has the same sort of problem as the spinor representation of $su(2) = so(3)$, which was not a representation of $SO(3)$, but only of its double cover $SU (2) = Spin(3)$"
\end{frame}


\section{The Groenwold-van Hove no-go theorem}
\begin{frame}{Main challenges}
	The issue lies here: how can we quantize polynomial functions on phase space with a degree greater than two?
\begin{itemize}
	\item Operator Ordering Ambiguities: Ordering matters.
	\item Inconsistency with Poisson Bracket Relation.
	\item Lowest Order Approximation.
	\item Limited Lie Algebra Representation.
\end{itemize}
And from the physics point of view, different ways of ordering the P and Q operators will lead to different operators $O_{f}$ for the same function $f$, with physically different observables.
\end{frame}

\begin{frame}{Main challenges}
For polynomials of degree greater than two there is no possible way to do this consistent with the following relation:
\begin{displaymath}
  Q_{\{f,g\}} = -\frac{i}{\hbar}\left[ O_f,O_g \right].
\end{displaymath}
"Whatever method one devises for quantizing higher-degree polynomials, it can only satisfy that relation to lowest order in $\hbar$, and there will be higher-order corrections, which depend upon one's choice of quantization scheme."
\end{frame}

\begin{frame}{Groenwold-van Hove no-go theorem}
\begin{theorem}
There is no map $f \rightarrow O_{f}$ from polynomials on $\mathbf{R}^2$ to self-adjoint operators on $L^2(\mathbf{R})$ satisfying
\begin{displaymath}
  O_{\{f,g\}} = -\frac{i}{\hbar}\left[O_f, O_g\right]
\end{displaymath}
and
\begin{displaymath}
  O_{p} = P, O_q=Q,
\end{displaymath}
or any Lie subalgebra of the functions on $\mathbf{R}^2$ for which the subalgebra of polynomials of degree less than or equal to two is a proper subalgebra.

\end{theorem}
\end{frame}

\section{Canonical quantization in $d$ dimensions}
\begin{frame}{Generalization}
Moving on to $d$ dimensions, we have
\begin{displaymath}
  \Gamma'_{S}(q_j)=-iQ_j, \Gamma'_{S}(p_j)=-iP_j,
\end{displaymath}
wich satisfy the Heisenberg relations
\begin{displaymath}
  \left[ Q_j, P_k\right] = i\delta_{jk}
\end{displaymath}
And for quadratic polynomials
\begin{displaymath}
  \Gamma'_{S}(q_j q_k)= -iQ_j Q_k, \Gamma'_{S}(p_j p_k)= -iP_j P_k
\end{displaymath}
\begin{displaymath}
  \Gamma'_{S}(q_j p_k)= -i\frac{i}{2}\left(Q_jP_k+P_kQ_j\right)
\end{displaymath}
\end{frame}

\section{Quantization and Symmetries}{}
\begin{frame}{Example: Angular Momentum}
"The observables that commute with the Hamiltonian operator H will make up a Lie algebra of symmetries of the quantum system and will take energy eigenstates to energy eigenstates of the same energy."

\textbf{Example}: The group SO(3).

The following operators provide a basis for the Lie algebra representation
\begin{displaymath}
  -i\left( Q_2 P_3 -Q_3P_2\right),-i\left( Q_3 P_1 -Q_1P_3\right), -i\left( Q_1 P_2 -Q_2 P_1\right)
\end{displaymath}
\end{frame}

\section{General Ways of Quantization}
\begin{frame}{Feynman Path Integral}
The key ideas are:
\begin{itemize}
	\item The quantum amplitude is calculated by summing over all possible paths a system can take between two states.
	\item It naturally incorporates the principle of least action from classical mechanics.
	\item The method is particularly useful in quantum field theory and for systems with many degrees of freedom.
\end{itemize}
\end{frame}

\end{document}