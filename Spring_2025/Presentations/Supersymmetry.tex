%%%%%%%%%%%%%%%%%%%%%%%%%%%%%%%%%%%%
%%%%% Supersymmetry
%%%%%%%%%%%%%%%%%%%%%%%%%%%%%%%%%%%%
%%%%%%%%%%%%%%%%%%%%%%%%%%%%%%%%%%%%
%%%%% Advanced Mathematical Methods
%%%%%%%%%%%%%%%%%%%%%%%%%%%%%%%%%%%%
\documentclass[12pt]{beamer}
\usepackage{amsmath}
%\usepackage{charter}
\usetheme[progressbar=frametitle]{metropolis}
%\usefonttheme{professionalfonts} % required for mathspec
%\usepackage{mathspec}

\title{Supersymmetry}
\author{Emmanuel Flores}
\institute{Advanced Mathematical Methods, \\Tufts University}

\begin{document}
\maketitle
%%%%%%%%%%%%%%%%%%%%%%%%%
\begin{frame}{Contents}
	\setbeamertemplate{section in toc}[sections numbered]
	\tableofcontents
\end{frame}
%%%%%%%%%%%%%%%%%%%%%%%%%
\section[Motivation]{Motivation}
\begin{frame}{Combining Bosonic and Fermionics Systems}
By taking tensor products ($\otimes$) of bosonic and fermionic systems, the operators (bosonic of fermionic) will  continue to act on the combined systems.

And the idea is to consider some very special operators:
\begin{itemize}
	\item They appear to mix bosonic and fermionic systems.
	\item They commute with the hamiltonian $H$.
\end{itemize}
\end{frame}
%%%%%%%%%%%%%%%%%%%%%%%%%
\section[The supersymmetric oscillator]{The supersymmetric oscillator}
\begin{frame}{Bosonic Harmonic Oscillator}
\begin{definition}
The Hamiltonian of a bosonic harmonic oscillator in $d$-dimensions, with state space $\mathcal{F}_d$ is given by:
\begin{displaymath}
H = \frac{1}{2}\hbar\omega\sum_{j=1}^{d} (a^{\dagger}_{B_j}a_{B_j} +a_{B_j}a^{\dagger}_{B_j}), 
\end{displaymath}
or 
\begin{displaymath}
H  = \sum_{j=1}^{d}(N_{B_j}+\frac{1}{2})\hbar\omega
\end{displaymath}
where $N_{B_j}$ has eigenvalues $n_{B_j}=0,1,2,\dots$
\end{definition}
\end{frame}
%%%%%%%%%%%%%%%%%%%%%%%%%
\begin{frame}{Fermionic Harmonic Oscillator}
\begin{definition}
The Hamiltonian of a fermionic harmonic oscillator in $d$-dimensions, with state space $\mathcal{F}^{\dagger}_d$ is given by:
\begin{displaymath}
H = \frac{1}{2}\hbar\omega \sum_{j=1}^{d} (a^{\dagger}_{F_j}a_{F_j} - a_{B_j}a^{\dagger}_{B_j}), 
\end{displaymath}
or 
\begin{displaymath}
H  = \sum_{j=1}^{d}(N_{F_j}-\frac{1}{2})\hbar\omega
\end{displaymath}
where $N_{B_j}$ has eigenvalues $n_{B_j}=0,1$
\end{definition}
\end{frame}
%%%%%%%%%%%%%%%%%%%%%%%%%
\begin{frame}{Definition of the Full System}
By taking the state spaces $\mathcal{F}_d$ and $\mathcal{F}^{\dagger}_d$ we can make the new state
\begin{displaymath}
  \mathcal{H} = \mathcal{F}_d\otimes\mathcal{F}^{\dagger}_d
\end{displaymath}
with a corresponding Hamiltonian given by
\begin{displaymath}
  H = \sum_{j=1}^{d}(N_{B_j}+N_{F_j})\hbar\omega,
\end{displaymath}
and the lowest energy state has energy $|0\rangle$ has energy $0$.
\end{frame}
%%%%%%%%%%%%%%%%%%%%%%%%%
\begin{frame}{Going back to 1D}
If we just look at $d=1$, we have
\begin{displaymath}
  H = \left( N_{B} + N_{F}\right)\hbar\omega,
\end{displaymath}
where the eigenvalue problem reads
\begin{displaymath}
  H| n_B, n_F \rangle = \left( n_B + n_F\right)\hbar\omega.
\end{displaymath}
Some facts about the system:
\begin{itemize}
	\item The ground state is uniquel $|0,0\rangle$
	\item All the nonzero energy states come in the following pairs $|n,0\rangle$, $|n-1,1\rangle$, both with energy $n\hbar\omega$
\end{itemize}
\end{frame}
%%%%%%%%%%%%%%%%%%%%%%%%%
\begin{frame}{Looking for Symmetries}
Degeneracy usually(not always) implies some kind of symmetries, a.k.a the existence of some operator that will commute with the Hamiltonian.
By looking at operators that $|n,0\rangle \longleftrightarrow |n-1,1\rangle$ we define
\begin{displaymath}
  Q_{+}=a_B a_{f}^{\dagger}, Q_{-}=a_B^{\dagger} a_{f}
\end{displaymath}
which by themselves are not self-adjoint, but are each other adjoints.
\end{frame}
%%%%%%%%%%%%%%%%%%%%%%%%%
\begin{frame}{Looking for symmetries(2)}
We can prove that 
\begin{displaymath}
  Q_{+}^2 = Q_{-}^2=0,
\end{displaymath}
together with
\begin{displaymath}
  \left(Q_{+}+Q_{-}\right)^2 = \left[ Q_{+},Q_{-}\right]_{+} = H,
\end{displaymath}
therefore, we could easily work with
\begin{displaymath}
  Q_{1} = Q_{+}+Q_{-}, Q_{2}=\frac{1}{i}\left(Q_{+}-Q_{-}\right)
\end{displaymath}
and these new operators will satisty
\begin{displaymath}
  \left[Q_1,Q_2\right]_{+}=0,Q_{1}^2=Q_{2}^2=H
\end{displaymath}
\end{frame}
%%%%%%%%%%%%%%%%%%%%%%%%%
\begin{frame}{Looking for symmetries(3)}
Using the previous result, $H$ can be expressed as the square of some operator, $Q_{+} +Q_{-}$, which implies that the eigenvalues are nonnegative. And even more, the eigenstates will be degenerate and will come in pairs
\begin{displaymath}
  |\psi\rangle , Q_{+} +Q_{-}|\psi\rangle.
\end{displaymath}
Therefore, to find the state of zero energy, we can look for the sollowing solutions
\begin{displaymath}
  Q_1|0\rangle=0, Q_2|0\rangle=0,
\end{displaymath}
and this result is similar to what happens with the usual BHO, in which the lowest energy state in various representations can be found by looking at solutions to $a|0\rangle=0$.
\end{frame}
%%%%%%%%%%%%%%%%%%%%%%%%%
\begin{frame}{Physical Realization}
It turns out that there exista a physical system with this exact behavior: A charged particle confined to a plane, coupled to a magnetic field perpendicular to the plane.
\begin{itemize}
	\item In this system, the equally spaced energy lavels are called \textbf{Landau levels}.
	\item If the particle has spin, there's an external term in the hamiltonian, $-\mu\mathbf{S}\cdot\mathbf{B}$.
\end{itemize}
And in the case of a gyromagnetic ratio $g=2$, the match up between the SSHO and this system is exact, they present the same pattern of energy levels.
\end{frame}
%%%%%%%%%%%%%%%%%%%%%%%%%
\section[Supersymmetric quantum mechanics with a superpotential]{Supersymmetric quantum mechanics with a superpotential}
\begin{frame}{Looking for generalizations}
Let's lift the SSHO to a wider class of potentials, while preserving the suersymmetry. In analogy with the creating and anhiltion bosonic operators we can introduce a superpotential $W(q)$ and define
\begin{displaymath}
a_{B} = \frac{1}{\sqrt{2}}\left( W^\prime(Q) + iP\right), a_{B}^{\dagger} = \frac{1}{\sqrt{2}}\left( W^\prime(Q) - iP\right),
\end{displaymath}
where $W^\prime(Q)$ is the multiplication operator $W^{\prime}(q)$ in the Schr�dinger position space representation on functions of q.
\end{frame}
%%%%%%%%%%%%%%%%%%%%%%%%%
\begin{frame}{Looking for generalizations(2)}
By taking $W(q) = \frac{q^2}{2}$ together with
\begin{displaymath}
Q_{+} = a_{B}a_{F}^{\dagger}, Q_{-} = a_{B}^{\dagger}a_{F}
\end{displaymath}
we can prove that
\begin{displaymath}
Q_{+}^2 = Q_{-}^2 = 0.
\end{displaymath}
And even more we find
\begin{displaymath}
H = (Q_{+} + Q_{-})^2,
\end{displaymath}
\begin{displaymath}
\implies H =\frac{1}{2}\left( W^\prime(Q)^2 + P^2\right) + \frac{1}{2}\left( i[P, W^\prime(Q)]\right)\sigma_3
\end{displaymath}
\end{frame}
%%%%%%%%%%%%%%%%%%%%%%%%%
\begin{frame}{Looking for generalizations(3)}
But $iP$ is the operator corresponding to infinitesimal translations in Q, thus
\begin{displaymath}
i[P,W^\prime(Q)]=W^{\prime\prime},
\end{displaymath}
which implies thath
\begin{displaymath}
  H = \frac{1}{2}\left( W^{\prime}(Q)^2 + P^2\right) + \frac{1}{2}W^{\prime\prime}(Q)\sigma_3,
\end{displaymath}
therefore, by taking different choices of $W$ we have a large class of quantum systems that can be used as toy models to investigate properties of qantum ground states.
\end{frame}
%%%%%%%%%%%%%%%%%%%%%%%%%
\begin{frame}{Looking for generalizations(4)}
All the previous class of models have the same state space
\begin{displaymath}
\mathcal{H} = \mathcal{H}_B\otimes\mathcal{F}_d^\dagger = L^2(\mathbf{R})\otimes \mathbf{C}^2,
\end{displaymath}
and all the energies eigen alues will be nonnegative, and the eigenvectors with positive energy will occur in pairs
\begin{displaymath}
|\psi\rangle, \left(Q_{+} + Q_{-}\right)|\psi\rangle.
\end{displaymath}
\end{frame}
%%%%%%%%%%%%%%%%%%%%%%%%%
\begin{frame}{Breaking the symmetry(1)}
If the lowest energy is not unique, and there exist a symmetry group that acts non-trivually on the space of lowest energy states, then the symmetry is said to be \textbf{spontaneously broken}.

In the context of supersymmetric quantum mechanics(by thinking in terms of Lie superalgebras) one calls $Q_1$ the generator of the action of a supersymmetry, with $H$ invariant under the supersymmetry in the sense that $[Q_1, H]=0$.
\end{frame}
%%%%%%%%%%%%%%%%%%%%%%%%%
\begin{frame}{Breaking the symmetry(2)}
And now we can ask the question: how the supersymmetry acts on the lowest energy state? The answer depends on wheter or not we can find solutions to 
\begin{displaymath}
  \left(Q_{+} + Q_{-}\right)|0\rangle=Q_1|0\rangle = 0.
\end{displaymath}
And if such solution exists, we say that the ground state is \textbf{invariant under the supersymmetry}. 
On the other hand, if this does not happen, then $Q_1$ will take a lowest energy state to another energy state, and in this case we say that the system has \textbf{spontaneously broken supersymmetry}.
\end{frame}
%%%%%%%%%%%%%%%%%%%%%%%%%
\begin{frame}{Breaking the symmetry(3)}
Appaarently the question of whether a given supersymmetric theory has its supersymmetry spontaneously broken or not is of great interest in more sophisticated theories.
There's this hope of making contact with the real worl via theories with such mechanism, i.e. where the supersymmetry is spontaneoulsy broken.
\end{frame}
%%%%%%%%%%%%%%%%%%%%%%%%%
\section[Supersymmetric quantum mechanics and differential forms]{Supersymmetric quantum mechanics and differential forms}
%%%%%%%%%%%%%%%%%%%%%%%%%
\begin{frame}{Going back to d-Dimensions}
By considering supersymmetric quantum mechanics in the case of $d$ degrees of freedom in the Schrodinger representation, the state space is described by
\begin{displaymath}
  \mathcal{H} = L^{2}(\mathbf{R}^d)\otimes\Lambda^{*}(\mathbf{C}^d),
\end{displaymath}
which is the tensor product of complex valued functions on $\mathbf{R}^d$ and anticummuting functions on $\mathbf{C}^d$.
\end{frame}
%%%%%%%%%%%%%%%%%%%%%%%%%
\begin{frame}{Differential forms}
It turns out that the space $\mathcal{H}$ is the space of complex-valued differential forms on $\mathbf{R}^d$, often denotes as $\Omega^*(\mathbf{R}^d)$.

In this space, we have an operator $d$ whose square is zero, which is called de Rham differential, and can be used to write the Laplacian operator on differential forms
\begin{displaymath}
  \square = (d+\delta)^2,
\end{displaymath}
where $\delta$ is the adjoint of $d$.
\end{frame}
%%%%%%%%%%%%%%%%%%%%%%%%%
\begin{frame}{Differential Forms(2)}
In this context, the supersymmetric quantum systems we've been studied corresponds to
\begin{displaymath}
Q_{+} = e^{-W(q)}de^{w(q)}, Q_{-}=e^{W(q)}\delta e^{-W(q)}
\end{displaymath}
\end{frame}
\end{document}