%%%%%%%%%%%%%%%%%%%%%%%%%%%%%%%%%%%%
%%%%% Supersymmetry
%%%%%%%%%%%%%%%%%%%%%%%%%%%%%%%%%%%%
%%%%%%%%%%%%%%%%%%%%%%%%%%%%%%%%%%%%
%%%%% Advanced Mathematical Methods
%%%%%%%%%%%%%%%%%%%%%%%%%%%%%%%%%%%%
\documentclass[12pt]{beamer}
\usepackage{amsmath}
%\usepackage{charter}
\usetheme[progressbar=frametitle]{metropolis}
%\usefonttheme{professionalfonts} % required for mathspec
%\usepackage{mathspec}

\title{Supersymmetry}
\author{Emmanuel Flores}
\institute{Advanced Mathematical Methods, \\Tufts University}

\begin{document}
\maketitle
%%%%%%%%%%%%%%%%%%%%%%%%%
\begin{frame}{Contents}
	\setbeamertemplate{section in toc}[sections numbered]
	\tableofcontents
\end{frame}
%%%%%%%%%%%%%%%%%%%%%%%%%
\section[Motivation]{Motivation}
\begin{frame}{Combining Bosonic and Fermionics Systems}
By taking tensor products ($\otimes$) of bosonic and fermionic systems, the operators (bosonic of fermionic) will  continue to act on the combined systems.

And the idea is to consider some very special operators:
\begin{itemize}
	\item They appear to mix bosonic and fermionic systems.
	\item They commute with the hamiltonian $H$.
\end{itemize}
\end{frame}
%%%%%%%%%%%%%%%%%%%%%%%%%
\section[The supersymmetric oscillator]{The supersymmetric oscillator}
\begin{frame}{Bosonic Harmonic Oscillator}
\begin{definition}
The Hamiltonian of a bosonic harmonic oscillator in $d$-dimensions, with state space $\mathcal{F}_d$ is given by:
\begin{displaymath}
H = \frac{1}{2}\hbar\omega\sum_{j=1}^{d} (a^{\dagger}_{B_j}a_{B_j} +a_{B_j}a^{\dagger}_{B_j}), 
\end{displaymath}
or 
\begin{displaymath}
H  = \sum_{j=1}^{d}(N_{B_j}+\frac{1}{2})\hbar\omega
\end{displaymath}
where $N_{B_j}$ has eigenvalues $n_{B_j}=0,1,2,\dots$
\end{definition}
\end{frame}
%%%%%%%%%%%%%%%%%%%%%%%%%
\begin{frame}{Fermionic Harmonic Oscillator}
\begin{definition}
The Hamiltonian of a fermionic harmonic oscillator in $d$-dimensions, with state space $\mathcal{F}^{\dagger}_d$ is given by:
\begin{displaymath}
H = \frac{1}{2}\hbar\omega \sum_{j=1}^{d} (a^{\dagger}_{F_j}a_{F_j} - a_{B_j}a^{\dagger}_{B_j}), 
\end{displaymath}
or 
\begin{displaymath}
H  = \sum_{j=1}^{d}(N_{F_j}-\frac{1}{2})\hbar\omega
\end{displaymath}
where $N_{B_j}$ has eigenvalues $n_{B_j}=0,1$
\end{definition}
\end{frame}
%%%%%%%%%%%%%%%%%%%%%%%%%
\begin{frame}{Definition of the Full System}
By taking the state spaces $\mathcal{F}_d$ and $\mathcal{F}^{\dagger}_d$ we can make the new state
\begin{displaymath}
  \mathcal{H} = \mathcal{F}_d\otimes\mathcal{F}^{\dagger}_d
\end{displaymath}
with a corresponding Hamiltonian given by
\begin{displaymath}
  H = \sum_{j=1}^{d}(N_{B_j}+N_{F_j})\hbar\omega,
\end{displaymath}
and the lowest energy state has energy $|0\rangle$ has energy $0$.
\end{frame}

\section[Supersymmetric quantum mechanics with a superpotential]{Supersymmetric quantum mechanics with a superpotential}
\begin{frame}
	
\end{frame}

\section[Supersymmetric quantum mechanics and differential forms]{Supersymmetric quantum mechanics and differential forms}
\end{document}