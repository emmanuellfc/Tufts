%%%%%%%%%%%%%%%%%%%%%%%%%%%%%%%%%%% Format and Stuff
\documentclass[11pt]{article}
\usepackage{geometry}
\geometry{
 a4paper,
 total={170mm,257mm},
 left=20mm,
 top=20mm,
 }
\usepackage{amsmath}
\usepackage{amssymb}
\usepackage{breqn}
\usepackage[sfdefault]{merriweather}
\usepackage[T1]{fontenc}
\newtheorem{definition}{Definition}
\newtheorem{problem}{Problem}
\newtheorem{proof}{Proof}

%%%%%%%%%%%%%%%%%%%%%%%%%%%%%%%%%%% Starting Documment
\title{GR-HW-01}
\author{J Emmanuel Flores}
\date{\today}

\begin{document}

\maketitle

\begin{problem}

\end{problem}
\begin{proof}

\end{proof}



\pagebreak
\begin{problem}[Two-Body Problem]
By performing a Lorentz transformation, show that $T_{\mu\nu}B^\mu A^\nu$ is a Lorentz scalar
\end{problem}
\begin{proof}
By definition a Lorentz scalar is an entity for which the following relationship holds
\begin{equation}
  \bar{T}_{\mu\nu}\bar{B}^\mu \bar{A}^\nu = T_{\mu\nu}B^\mu A^\nu,
\end{equation}
this is, it has the same form under a Lorentz transformation. So let's proceed with the proof. By definition, we have the following
\begin{equation}
  \bar{T}_{\mu\nu} = \Lambda_{\mu}^{\alpha_1}\Lambda_{\nu}^{\alpha_2}T_{\alpha_1 \alpha_2}
\end{equation}
and as well for the other two tensors of order one
\begin{equation}
  \bar{B}^\mu = \Lambda^{\mu}_{\alpha_3}B^{\alpha_3},\ \   \bar{A}^\nu = \Lambda^{\nu}_{\alpha_4}A^{\alpha_4}, 
\end{equation}
and from this we have that
\begin{equation}
  \bar{T}_{\mu\nu}\bar{B}^\mu \bar{A}^\nu = \Lambda_{\mu}^{\alpha_1}\Lambda_{\nu}^{\alpha_2}\Lambda^{\mu}_{\alpha_3}\Lambda^{\nu}_{\alpha_4}T_{\alpha_1 \alpha_2}B^{\alpha_3}A^{\alpha_4}.
\end{equation}
After rearangin the terms and using the fact that
\begin{equation}
  \Lambda_{\mu}^{\alpha_1}\Lambda^{\mu}_{\alpha_3} = \delta^{\alpha_1}_{\alpha_3}, \ \ \Lambda_{\nu}^{\alpha_2}\Lambda^{\nu}_{\alpha_4}= \delta^{\alpha_2}_{\alpha_4},
\end{equation}
then we have 
\begin{equation}
  \bar{T}_{\mu\nu}\bar{B}^\mu \bar{A}^\nu = \delta^{\alpha_1}_{\alpha_3}\delta^{\alpha_2}_{\alpha_4}T_{\alpha_1 \alpha_2}B^{\alpha_3}A^{\alpha_4},
\end{equation}
and then the RHS of the previous equation takes the form
\begin{equation}
  T_{\alpha_1 \alpha_2}B^{\alpha_1}A^{\alpha_2},
\end{equation}
but since those are dummy indices, we can change them to make the equation prettier, leaving us with
\begin{equation}
  \bar{T}_{\mu\nu}\bar{B}^\mu \bar{A}^\nu = T_{\mu\nu}B^\mu A^\nu,
\end{equation}
just as we wanted.
\end{proof}

\end{document}
